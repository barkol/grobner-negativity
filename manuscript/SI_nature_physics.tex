% Supplementary Information - Nature Physics format
\documentclass[11pt]{article}

\usepackage[utf8]{inputenc}
\usepackage[T1]{fontenc}
\usepackage{graphicx}
\usepackage{amsmath,amssymb,amsthm}
\usepackage{bm}

% Theorem environments
\newtheorem{theorem}{Theorem}[section]
\newtheorem{lemma}[theorem]{Lemma}
\newtheorem{proposition}[theorem]{Proposition}
\newtheorem{corollary}[theorem]{Corollary}
\theoremstyle{definition}
\newtheorem{definition}[theorem]{Definition}
\newtheorem{example}[theorem]{Example}
\theoremstyle{remark}
\newtheorem{remark}[theorem]{Remark}
\usepackage{booktabs}
\usepackage[colorlinks=true,linkcolor=blue,citecolor=blue,urlcolor=blue]{hyperref}
\usepackage[margin=2.5cm]{geometry}
\usepackage{tikz}
\usepackage{pgfplots}
\pgfplotsset{compat=1.17}
\usepackage{quantikz}
\usepackage{multirow}

\begin{document}

\begin{center}
{\Large\bfseries Supplementary Information}\\[0.5em]
{\large Spin chirality across quantum state copies detects hidden entanglement}\\[1em]
Patrycja Tulewicz and Karol Bartkiewicz
\end{center}

\vspace{1em}

\setcounter{section}{0}
\renewcommand{\thesection}{S\arabic{section}}
\renewcommand{\theequation}{S\arabic{equation}}
\renewcommand{\thefigure}{S\arabic{figure}}
\renewcommand{\thetable}{S\arabic{table}}

\tableofcontents
\newpage

%==============================================================================
\section{Measurement circuits}
\label{sec:circuits}
%==============================================================================

This section presents quantum circuits for measuring partial transpose moments $\mu_k$ and purity invariants $I_k$. All measurements follow the Hadamard test structure: prepare $n$ state copies, apply controlled permutation operators, and extract the invariant from the ancilla measurement probability via $X = 2p_0 - 1$.

\subsection{Partial transpose moments}

The partial transpose moments $\mu_k = \mathrm{Tr}[(\rho^{T_A})^k]$ use a cycle--anticycle structure:
\begin{itemize}
\item \textbf{Cycle block}: Cyclic permutation on subsystem A
\item \textbf{Anticycle block}: Reverse cyclic permutation on subsystem B
\end{itemize}

\begin{figure}[htb!]
\centering
\begin{quantikz}[row sep=0.5cm, column sep=0.4cm]
    \lstick{$A_1$} & \qw & \swap{1} & \qw & \qw & \qw \\
    \lstick{$A_2$} & \qw & \targX{} & \qw & \qw & \qw \\
    \lstick{$B_1$} & \qw & \qw & \swap{1} & \qw & \qw \\
    \lstick{$B_2$} & \qw & \qw & \targX{} & \qw & \qw \\
    \lstick{Anc} & \gate{H} & \ctrl{-4} & \ctrl{-2} & \gate{H} & \meter{}
\end{quantikz}
\caption{Circuit $\mathcal{C}_{\mu_2}$ for the second partial transpose moment. Outcome: $\mu_2 = 2p_0 - 1$.}
\label{fig:mu2_circuit}
\end{figure}

\begin{figure}[htb!]
\centering
\begin{quantikz}[row sep=0.4cm, column sep=0.3cm]
    \lstick{$A_1$} & \qw & \swap{1} & \qw & \qw & \qw & \qw & \qw \\
    \lstick{$A_2$} & \qw & \targX{} & \swap{1} & \qw & \qw & \qw & \qw \\
    \lstick{$A_3$} & \qw & \qw & \targX{} & \qw & \qw & \qw & \qw \\
    \lstick{$B_1$} & \qw & \qw & \qw & \qw & \swap{1} & \qw & \qw \\
    \lstick{$B_2$} & \qw & \qw & \qw & \swap{1} & \targX{} & \qw & \qw \\
    \lstick{$B_3$} & \qw & \qw & \qw & \targX{} & \qw & \qw & \qw \\
    \lstick{Anc} & \gate{H} & \ctrl{-6} & \ctrl{-5} & \ctrl{-2} & \ctrl{-3} & \gate{H} & \meter{}
\end{quantikz}
\caption{Circuit $\mathcal{C}_{\mu_3}$ for the third partial transpose moment.}
\label{fig:mu3_circuit}
\end{figure}

\begin{figure}[htb!]
\centering
\begin{quantikz}[row sep=0.35cm, column sep=0.25cm]
    \lstick{$A_1$} & \qw & \swap{1} & \qw & \qw & \qw & \qw & \qw & \qw & \qw \\
    \lstick{$A_2$} & \qw & \targX{} & \swap{1} & \qw & \qw & \qw & \qw & \qw & \qw \\
    \lstick{$A_3$} & \qw & \qw & \targX{} & \swap{1} & \qw & \qw & \qw & \qw & \qw \\
    \lstick{$A_4$} & \qw & \qw & \qw & \targX{} & \qw & \qw & \qw & \qw & \qw \\
    \lstick{$B_1$} & \qw & \qw & \qw & \qw & \qw & \qw & \swap{1} & \qw & \qw \\
    \lstick{$B_2$} & \qw & \qw & \qw & \qw & \qw & \swap{1} & \targX{} & \qw & \qw \\
    \lstick{$B_3$} & \qw & \qw & \qw & \qw & \swap{1} & \targX{} & \qw & \qw & \qw \\
    \lstick{$B_4$} & \qw & \qw & \qw & \qw & \targX{} & \qw & \qw & \qw & \qw \\
    \lstick{Anc} & \gate{H} & \ctrl{-8} & \ctrl{-7} & \ctrl{-6} & \ctrl{-2} & \ctrl{-3} & \ctrl{-4} & \gate{H} & \meter{}
\end{quantikz}
\caption{Circuit $\mathcal{C}_{\mu_4}$ for the fourth partial transpose moment.}
\label{fig:mu4_circuit}
\end{figure}

\subsection{Purity invariants}

The purity $I_2 = \mathrm{Tr}[\rho^2]$ uses symmetric cyclic permutations on both subsystems. A key identity simplifies measurement: $\mu_2 = I_2$ for all bipartite states. This means the same circuit measures both second-order invariants.

\subsection{Resource summary}

\begin{table}[htb!]
\centering
\caption{Circuit resources for invariant measurements.}
\label{tab:circuit_resources}
\begin{tabular}{@{}lcccc@{}}
\toprule
Invariant & Copies & Qubits ($2{\times}2$) & Qubits ($2{\times}3$) & CSWAPs \\
\midrule
$I_2 = \mu_2$ & 2 & 5 & 7 & 2 \\
$\mu_3$ & 3 & 7 & 10 & 4 \\
$\mu_4$ & 4 & 9 & 13 & 6 \\
\bottomrule
\end{tabular}
\end{table}

%==============================================================================
\section{Gr\"obner basis derivations}
\label{sec:grobner}
%==============================================================================

\subsection{Degeneracy conditions for two-qubit systems}

For a two-qubit system, the partial transpose $\rho^{T_A}$ has four eigenvalues $\{\lambda_1, \lambda_2, \lambda_3, \lambda_4\}$ with $\sum_i \lambda_i = 1$. The moments $\mu_k = \sum_i \lambda_i^k$ relate to elementary symmetric polynomials via Newton--Girard identities.

The discriminant $\mathcal{D}(\mu_2, \mu_3, \mu_4)$ vanishes if and only if at least two eigenvalues coincide. This degree-6 polynomial is:
\begin{align}
576\,\mathcal{D} &= 72\mu_2^6 - 684\mu_2^5 + 1536\mu_2^4\mu_3 - 720\mu_2^4\mu_4 + 1209\mu_2^4 \nonumber\\
&\quad - 544\mu_2^3\mu_3^2 - 4168\mu_2^3\mu_3 + 3096\mu_2^3\mu_4 - 796\mu_2^3 + \cdots
\label{eq:discriminant}
\end{align}

For specific degeneracy types:

\textbf{Triple degeneracy} $(\alpha, \alpha, \alpha, \beta)$:
\begin{equation}
\mathcal{G}_2 = 16\mu_2^3 - 39\mu_2^2 + 72\mu_2\mu_3 + 12\mu_2 - 48\mu_3^2 - 12\mu_3 - 1 = 0.
\end{equation}

\textbf{Two-pair degeneracy} $(\alpha, \alpha, \beta, \beta)$:
\begin{equation}
\mathcal{G}_1 = 6\mu_2 - 8\mu_3 - 1 = 0.
\end{equation}

Both conditions involve only $\mu_2$ and $\mu_3$, not $\mu_4$---the degeneracy type can be determined from two measurements.

\subsection{Negativity formulas}

For states satisfying $\mathcal{G}_2 = 0$ (triple degeneracy):
\begin{equation}
\mathcal{N}_{\text{triple}} = \max\left\{0, \frac{\sqrt{12\mu_2 - 3} - 1}{4}\right\}.
\end{equation}

For states satisfying $\mathcal{G}_1 = 0$ and $\mathcal{D} = 0$ (two-pair degeneracy):
\begin{equation}
\mathcal{N}_{\text{two-pair}} = \max\left\{0, \frac{\sqrt{4\mu_2 - 1} - 1}{2}\right\}.
\end{equation}

For generic states ($\mathcal{D} \neq 0$), eigenvalues are computed from the characteristic polynomial using Newton--Girard identities:
\begin{align}
e_1 &= 1, \quad e_2 = \tfrac{1}{2}(1 - \mu_2), \nonumber\\
e_3 &= \tfrac{1}{6}(1 - 3\mu_2 + 2\mu_3), \nonumber\\
e_4 &= \tfrac{1}{24}(1 - 6\mu_2 + 8\mu_3 + 3\mu_2^2 - 6\mu_4).
\end{align}

\subsection{Qubit--qutrit extensions}

For qubit--qutrit systems ($2 \times 3$), the partial transpose has six eigenvalues. Analogous conditions are:

\textbf{Quintuple degeneracy}:
\begin{equation}
\mathcal{G}_2^{(2\times3)} = 96\mu_2^3 - 93\mu_2^2 + 180\mu_2\mu_3 + 18\mu_2 - 180\mu_3^2 - 20\mu_3 - 1 = 0.
\end{equation}

\textbf{Two-triple degeneracy}:
\begin{equation}
\mathcal{G}_1^{(2\times3)} = 9\mu_2 - 18\mu_3 - 1 = 0.
\end{equation}

%==============================================================================
\section{Invariant identities}
\label{sec:identities}
%==============================================================================

\subsection{The identity \texorpdfstring{$\mu_2 = I_2$}{mu2 = I2}}

For any bipartite state $\rho$ on $\mathcal{H}_A \otimes \mathcal{H}_B$:

\textbf{Claim:} $\mathrm{Tr}[(\rho^{T_A})^2] = \mathrm{Tr}[\rho^2]$.

\textbf{Proof:} In the computational basis, $\rho = \sum_{ijkl} \rho_{ij,kl} |ij\rangle\langle kl|$. The partial transpose is $\rho^{T_A} = \sum_{ijkl} \rho_{kj,il} |ij\rangle\langle kl|$. Computing:
\begin{align}
\mathrm{Tr}[(\rho^{T_A})^2] &= \sum_{ijkl} \rho_{kj,il} \rho_{il,kj} = \sum_{ijkl} |\rho_{ij,kl}|^2 = \mathrm{Tr}[\rho^2]. \quad \square
\end{align}

This identity is dimension-independent and implies only one purity circuit is needed.

\subsection{Moment relations}
\label{subsec:moment_relations}

The partial transpose moments $\mu_k$ and purity moments $I_k$ admit a unified description as permutation traces on the $k$-copy Hilbert space:
\begin{align}
\mu_k &= \mathrm{Tr}\bigl[(\sigma_A^{-1} \otimes \sigma_B)\,\rho^{\otimes k}\bigr], \\
I_k &= \mathrm{Tr}\bigl[(\sigma_A \otimes \sigma_B)\,\rho^{\otimes k}\bigr],
\end{align}
where $\sigma = (12\cdots k)$ is the cyclic permutation on $k$ copies, $\sigma^{-1} = (k\cdots 21)$ its inverse, and subscripts $A$, $B$ denote action on the respective subsystems. Their difference is governed by a single operator:
\begin{equation}
C_k \equiv \mu_k - I_k = \mathrm{Tr}\bigl[(\Delta_A \otimes \sigma_B)\,\rho^{\otimes k}\bigr], \quad \Delta \equiv \sigma^{-1} - \sigma.
\end{equation}

For two-qubit systems, the cyclic permutation decomposes via adjacent SWAPs: $\sigma = S_{12}S_{23}\cdots S_{k-1,k}$ with $S_{mn} = \tfrac{1}{2}(I + \bm{\sigma}_m \cdot \bm{\sigma}_n)$. Defining $g_{mn} = \bm{\sigma}_m \cdot \bm{\sigma}_n = 2S_{mn} - I$, the key identity connecting SWAPs to chirality is:
\begin{equation}
[g_{ij},\; g_{jk}] = -16i\,\chi_{ijk},
\label{eq:comm_identity}
\end{equation}
where $\chi_{ijk} = \mathbf{S}_i \cdot (\mathbf{S}_j \times \mathbf{S}_k) = \tfrac{1}{8}\sum_{abc}\varepsilon_{abc}\,\sigma_i^a\sigma_j^b\sigma_k^c$ is the scalar spin chirality operator. Disjoint pairs commute: $[g_{ij}, g_{mn}] = 0$ when $\{i,j\} \cap \{m,n\} = \emptyset$.

These identities yield explicit chirality decompositions of $\Delta$ for each $k$:

\paragraph{$k = 2$:} $\sigma = \sigma^{-1} = S_{12}$, so $\Delta = 0$ and
\begin{equation}
\mu_2 = I_2.
\end{equation}

\paragraph{$k = 3$:} $\sigma = S_{12}S_{23}$, $\sigma^{-1} = S_{23}S_{12}$. The difference is a commutator:
\begin{equation}
\Delta = [S_{23}, S_{12}] = \tfrac{1}{4}[g_{23}, g_{12}] = 4i\,\chi_{123},
\end{equation}
giving
\begin{equation}
\mu_3 = I_3 + C_3, \quad C_3 = 4i\,\mathrm{Tr}\bigl[\chi_A\,\sigma_B\;\rho^{\otimes 3}\bigr].
\label{eq:mu3_chirality}
\end{equation}

\paragraph{$k = 4$:} $\sigma = S_{12}S_{23}S_{34}$, $\sigma^{-1} = S_{34}S_{23}S_{12}$. Expanding:
\begin{align}
8\Delta &= (I{+}g_{34})(I{+}g_{23})(I{+}g_{12}) - (I{+}g_{12})(I{+}g_{23})(I{+}g_{34}) \nonumber\\
&= [g_{23},g_{12}] + [g_{34},g_{23}] + g_{34}[g_{23},g_{12}] + g_{12}[g_{34},g_{23}] \nonumber\\
&= 16i\bigl\{(I{+}g_{34})\chi_{123} + (I{+}g_{12})\chi_{234}\bigr\} \nonumber\\
&= 32i\bigl(S_{34}\,\chi_{123} + S_{12}\,\chi_{234}\bigr),
\end{align}
where the third line uses $[g_{12}, g_{34}] = 0$ and $I + g_{mn} = 2S_{mn}$.

To simplify $\Omega = S_{34}\chi_{123} + S_{12}\chi_{234}$, we use $S_{mn} = \tfrac{1}{2}(I + g_{mn})$ and evaluate the products $g_{mn}\chi_{ijk}$ via Pauli algebra. When an index is shared (e.g., index 3 in $g_{34}\chi_{123}$), we use
\begin{equation}
\sigma_3^a \sigma_3^d = \delta_{ad} I + i\sum_e \varepsilon_{ade}\,\sigma_3^e
\end{equation}
and the Levi-Civita contraction identity $\sum_d \varepsilon_{bcd}\varepsilon_{ade} = \delta_{ba}\delta_{ce} - \delta_{be}\delta_{ca}$. This yields:
\begin{align}
g_{34}\chi_{123} &= \chi_{124} + \tfrac{i}{8}(g_{14}g_{23} - g_{13}g_{24}), \\
g_{12}\chi_{234} &= \chi_{134} + \tfrac{i}{8}(g_{13}g_{24} - g_{14}g_{23}).
\end{align}
The imaginary terms cancel upon addition, giving
\begin{equation}
g_{34}\chi_{123} + g_{12}\chi_{234} = \chi_{124} + \chi_{134}.
\end{equation}
Therefore:
\begin{align}
\Omega &= \tfrac{1}{2}(I + g_{34})\chi_{123} + \tfrac{1}{2}(I + g_{12})\chi_{234} \nonumber\\
&= \tfrac{1}{2}(\chi_{123} + \chi_{234}) + \tfrac{1}{2}(\chi_{124} + \chi_{134}) \nonumber\\
&= \tfrac{1}{2}\sum_{i<j<k} \chi_{ijk},
\end{align}
where the sum runs over all $\binom{4}{3} = 4$ chirality triples on four copies. The fourth-order correction is thus:
\begin{equation}
\mu_4 = I_4 + C_4, \quad C_4 = 2i\,\mathrm{Tr}\Bigl[\Bigl(\sum_{i<j<k}\chi_{ijk}^A\Bigr)\,\sigma_B\;\rho^{\otimes 4}\Bigr].
\label{eq:mu4_chirality}
\end{equation}

The physical content is that the difference between partial transpose moments and purity moments is entirely controlled by scalar spin chirality operators acting across different state copies. For $k = 3$, a single chirality $\chi_{123}$ appears. For $k = 4$, Pauli algebra shows that all four chirality triples $\chi_{123}$, $\chi_{124}$, $\chi_{134}$, and $\chi_{234}$ contribute equally, yielding the symmetric sum $\Omega = \tfrac{1}{2}\sum_{i<j<k}\chi_{ijk}$.

\subsection{Hermitian correlator form}
\label{subsec:hermitian_form}

The chirality decompositions~\eqref{eq:mu3_chirality}--\eqref{eq:mu4_chirality} involve the non-Hermitian cyclic permutation $\sigma_B$, making the reality of $C_k$ non-obvious. We now derive a manifestly Hermitian form that reveals $C_k$ as a chirality--chirality correlator.

\paragraph{Anti-Hermiticity of $\Delta$.} Since the cyclic permutation $\sigma$ is unitary with $\sigma^\dagger = \sigma^{-1}$:
\begin{equation}
\Delta^\dagger = (\sigma^{-1})^\dagger - \sigma^\dagger = \sigma - \sigma^{-1} = -\Delta.
\end{equation}
Thus $\Delta$ is anti-Hermitian.

\paragraph{Symmetric form.} Taking the complex conjugate of $C_k = \mathrm{Tr}[(\Delta_A \otimes \sigma_B)\,\rho^{\otimes k}]$ and using $\Delta_A^\dagger = -\Delta_A$, $\sigma_B^\dagger = \sigma_B^{-1}$:
\begin{equation}
C_k^* = \mathrm{Tr}\bigl[(-\Delta_A \otimes \sigma_B^{-1})\,\rho^{\otimes k}\bigr].
\end{equation}
Since $C_k$ is real ($C_k = C_k^*$), averaging gives:
\begin{equation}
C_k = \tfrac{1}{2}(C_k + C_k^*) = \tfrac{1}{2}\,\mathrm{Tr}\bigl[\Delta_A \otimes (\sigma_B - \sigma_B^{-1})\,\rho^{\otimes k}\bigr] = -\tfrac{1}{2}\,\mathrm{Tr}\bigl[(\Delta_A\,\Delta_B)\,\rho^{\otimes k}\bigr].
\label{eq:Ck_symmetric}
\end{equation}
The operator $\Delta_A\,\Delta_B$ is Hermitian: since $\Delta_A$ and $\Delta_B$ act on disjoint Hilbert spaces they commute, so $(\Delta_A\,\Delta_B)^\dagger = \Delta_B^\dagger\,\Delta_A^\dagger = (-\Delta_B)(-\Delta_A) = \Delta_A\,\Delta_B$.

\paragraph{Chirality correlator.} Substituting the chirality decomposition $\Delta = 4i\,\Omega_k$ where $\Omega_k$ is Hermitian ($\Omega_3 = \chi_{123}$, $\Omega_4 = \tfrac{1}{2}\sum_{i<j<k}\chi_{ijk}$):
\begin{equation}
\Delta_A\,\Delta_B = (4i\,\Omega_A)(4i\,\Omega_B) = -16\,\Omega_A\,\Omega_B.
\end{equation}
Inserting into equation~\eqref{eq:Ck_symmetric}:
\begin{equation}
C_k = 8\,\mathrm{Tr}\bigl[\Omega_A\,\Omega_B\;\rho^{\otimes k}\bigr]
\label{eq:Ck_hermitian}
\end{equation}
Explicitly:
\begin{align}
C_3 &= 8\,\mathrm{Tr}\bigl[\chi_A\,\chi_B\;\rho^{\otimes 3}\bigr], \label{eq:C3_hermitian}\\
C_4 &= 8\,\mathrm{Tr}\bigl[\Omega_A\,\Omega_B\;\rho^{\otimes 4}\bigr], \quad \Omega = \tfrac{1}{2}\sum_{i<j<k}\chi_{ijk}. \label{eq:C4_hermitian}
\end{align}
The chirality correction is thus a \emph{chirality--chirality correlator}: the Hermitian chirality operator on subsystem $A$ is correlated with the corresponding operator on subsystem $B$, measured across $k$ state copies. For product states both factors vanish independently (coplanar spin configurations), giving $C_k = 0$. For pure states, entanglement breaks this coplanarity on both subsystems simultaneously, generating $C_k \neq 0$. For mixed states, including mixed separable states, $C_k$ is generally non-zero due to cross-terms in the multi-copy expansion (see Section~\ref{subsec:completeness}).

%==============================================================================
\section{Multi-copy spin chirality: entanglement hidden from single-copy detection}
\label{sec:chirality}
%==============================================================================

\subsection{Conceptual framework}

The chirality correction $C_4 = \mu_4 - I_4$ reveals a fundamentally new type of quantum correlation that we term \emph{multi-copy chirality}. Unlike standard entanglement measures that characterize properties of individual quantum states, multi-copy chirality captures patterns that exist only in the joint statistics of multiple state copies---correlations that are completely invisible to any measurement on a single copy.

This phenomenon represents a new quantum paradox. Consider a Bell state $|\Phi^+\rangle = (|00\rangle + |11\rangle)/\sqrt{2}$. Any single copy is maximally mixed on each subsystem; local measurements reveal no structure. Even the standard entanglement---detectable via Bell inequalities or witnesses---is a property of individual copies. The multi-copy chirality is different: it emerges only when we examine how \emph{three or more copies} of the state relate to each other through controlled-SWAP operations.

\subsection{Singlet projector algebra}

The mathematical structure arises from singlet projectors acting across different copies. The singlet projector between qubits $i$ and $j$ is:
\begin{equation}
P^-_{ij} = \frac{1}{4} - \mathbf{S}_i \cdot \mathbf{S}_j = \frac{1}{4}(I - \sigma_i^x\sigma_j^x - \sigma_i^y\sigma_j^y - \sigma_i^z\sigma_j^z).
\end{equation}

When projectors share a common index, their commutator yields scalar spin chirality:
\begin{equation}
[P^-_{ij}, P^-_{jk}] = -i\,\chi_{ijk},
\end{equation}
where $\chi_{ijk} = \mathbf{S}_i \cdot (\mathbf{S}_j \times \mathbf{S}_k)$ is the scalar spin chirality operator.

\subsection{Multi-copy structure}

The crucial insight is that in our measurement protocol, indices $i$, $j$, $k$ refer to qubits from \emph{different state copies}. The measurement of $C_4$ requires four copies of the state, with qubits labelled:
\begin{itemize}
\item Copy 1: qubits $(A_1, B_1)$ at positions 0, 1
\item Copy 2: qubits $(A_2, B_2)$ at positions 2, 3
\item Copy 3: qubits $(A_3, B_3)$ at positions 4, 5
\item Copy 4: qubits $(A_4, B_4)$ at positions 6, 7
\end{itemize}

The moment relations (Section~\ref{subsec:moment_relations}) provide the rigorous mathematical connection. In the Hermitian correlator form (equation~\eqref{eq:C4_hermitian}), the fourth-order correction decomposes as:
\begin{equation}
C_4 = 8\,\mathrm{Tr}\bigl[\Omega_A\,\Omega_B\;\rho^{\otimes 4}\bigr],
\end{equation}
where $\Omega_A = \tfrac{1}{2}(\chi_{024} + \chi_{026} + \chi_{046} + \chi_{246})$ is a Hermitian operator on the $A$-qubits (using qubit position labels 0, 2, 4, 6 for copies 1--4), and $\Omega_B = \tfrac{1}{2}(\chi_{135} + \chi_{137} + \chi_{157} + \chi_{357})$ is its counterpart on the $B$-qubits (positions 1, 3, 5, 7). Here $\chi_{024} = \mathbf{S}_{A_1}\cdot(\mathbf{S}_{A_2}\times\mathbf{S}_{A_3})$ is the chirality of $A$-qubits from copies 1, 2, 3. The Pauli algebra derivation in Section~\ref{subsec:moment_relations} shows that all four chirality triples contribute equally with coefficient $1/2$. The correlator form makes it manifest that $C_4$ is real and measures the correlation of chirality patterns between the $A$ and $B$ subsystems.

\subsection{Physical interpretation}

The scalar spin chirality $\chi_{ijk} = \mathbf{S}_i \cdot (\mathbf{S}_j \times \mathbf{S}_k)$ measures whether three spins form a left-handed or right-handed configuration---their ``handedness'' or chirality. Geometrically, it equals the signed volume of the parallelepiped spanned by three spin vectors, or equivalently, half the solid angle subtended by their unit vectors on the Bloch sphere (Fig.~\ref{fig:chirality_geometry}).

\begin{figure}[htb!]
\centering
\includegraphics[width=0.9\textwidth]{figure_chirality_3d.pdf}
\caption{\textbf{Multi-copy correlations in the chirality correction.} \textbf{a}, Product states produce coplanar spin patterns with $\chi = 0$. \textbf{b}, Entangled states break coplanarity; parallelepiped volume equals $\chi \neq 0$. \textbf{c}, Solid angle $\omega = 2|\chi|$ on the Bloch sphere. \textbf{d}, Three-spin chirality $\chi_A$: triangular correlations between qubits from three copies. \textbf{e}, SWAP-weighted chirality $S_{34}\chi_{123}$: the SWAP extends three-copy chirality to a four-copy operator. \textbf{f}, The chirality correction $C_4 = \mu_4 - I_4$; for product states $C_4 = 0$, while entangled states generate $C_4 \neq 0$.}
\label{fig:chirality_geometry}
\end{figure}

In condensed matter physics, this same operator is the order parameter for:
\begin{itemize}
\item Chiral spin liquids (Kalmeyer--Laughlin states)
\item Anomalous Hall effect in frustrated magnets
\item Topological phases with broken time-reversal symmetry
\end{itemize}

The remarkable finding is that \emph{the same mathematical structure} appears in our multi-copy entanglement measurement, but with a radically different physical interpretation:
\begin{itemize}
\item In condensed matter: chirality between three physical spins in a lattice
\item In our protocol: chirality between qubits from three copies of a bipartite state
\end{itemize}

For product states, the multi-copy correlations remain ``coplanar''---the copies do not generate chiral patterns, and $\langle\chi_A\chi_B\rangle = 0$. For pure states, entanglement breaks this coplanarity, generating non-zero chirality that witnesses entanglement. For mixed separable states, however, $C_k$ is generally non-zero because $\mu_k$ and $I_k$ are nonlinear functions of $\rho$ that do not decompose linearly over convex combinations (see Section~\ref{subsec:completeness}).

\subsection{Comparison with standard nonclassicality}

Multi-copy chirality is conceptually distinct from established forms of quantum correlation:

\begin{table}[htb!]
\centering
\caption{Comparison of nonclassical correlation types.}
\label{tab:correlation_types}
\begin{tabular}{@{}lcc@{}}
\toprule
Correlation type & Copies required & Detection method \\
\midrule
Bell nonlocality & 1 & CHSH inequality \\
Steering & 1 & Steering inequalities \\
Entanglement & 1 & Witnesses, PPT \\
Multi-copy chirality & 3+ & $C_4 = \mu_4 - I_4 \neq 0$ \\
\bottomrule
\end{tabular}
\end{table}

The key distinction: standard correlations are properties of individual states, while multi-copy chirality is a property of the \emph{ensemble} of copies---a collective quantum signature.

\subsection{Chirality correction as entanglement indicator}

The chirality correction $C_4 = \mu_4 - I_4$ directly probes multi-copy chirality. Since both $\mu_4$ and $I_4$ are permutation traces measured via controlled-SWAP circuits, $C_4$ is obtained without additional measurement overhead.

For states expressed in the computational basis:
\begin{itemize}
\item $C_4 = 0$ for all product states $|ab\rangle$ (no chirality)
\item $C_4 = -3/4$ for all Bell states (maximum chirality magnitude)
\item $C_4(\theta) = -\sin^2\theta + \frac{1}{4}\sin^4\theta$ for parametrized states $|\psi(\theta)\rangle = \cos(\theta/2)|00\rangle + \sin(\theta/2)|11\rangle$
\end{itemize}

For pure states, $C_4 = 0$ at $\theta = 0$ (product state) and $|C_4|$ increases monotonically with entanglement, reaching $|C_4| = 3/4$ at $\theta = \pi/2$ (Bell state). The physical interpretation is that entangled superpositions generate non-coplanar multi-copy spin patterns that product states cannot produce.

\subsection{Properties of the chirality correction}
\label{subsec:completeness}

This section analyzes the mathematical properties of the chirality correction $C_4 = \mu_4 - I_4$.

\subsubsection{LU transformation properties}

\textbf{Key Result:} $C_4$ is LU-invariant for all states.

For a local unitary transformation $\rho \to (U_A \otimes U_B)\rho(U_A \otimes U_B)^\dagger$:
\begin{itemize}
\item $I_k = \mathrm{Tr}[\rho^k]$ is always LU-invariant (purity moments are preserved).
\item $\mu_k = \mathrm{Tr}[(\rho^{T_A})^k]$ is always LU-invariant, since $(U_A \otimes U_B)\rho(U_A \otimes U_B)^\dagger$ has partial transpose $(U_A^* \otimes U_B)\rho^{T_A}(U_A^T \otimes U_B^\dagger)$, which preserves eigenvalues.
\item Therefore $C_4 = \mu_4 - I_4$ is LU-invariant for all states.
\end{itemize}

This is an advantage over basis-dependent quantities: $C_4$ gives the same value regardless of the local basis chosen for measurement.

\subsubsection{Preliminary definitions and notation}

Before proceeding, we establish the mathematical objects and notation used throughout.

\paragraph{Setup.} Let $\rho$ be a density matrix (a positive semidefinite operator with unit trace) on the tensor product Hilbert space $\mathcal{H}_A \otimes \mathcal{H}_B$, where $\dim(\mathcal{H}_A) = d_A$ and $\dim(\mathcal{H}_B) = d_B$. We work in the computational basis $\{|i\rangle_A \otimes |j\rangle_B\}_{i=0,\ldots,d_A-1}^{j=0,\ldots,d_B-1}$.

\paragraph{Key operations.}

\begin{enumerate}
\item \textbf{Partial transpose.} The partial transpose with respect to subsystem $A$ is defined by its matrix elements:
\begin{equation}
(\rho^{T_A})_{ij,kl} = \rho_{kj,il}.
\end{equation}
\textit{Interpretation:} We transpose the ``$A$ indices'' (first and third) while leaving the ``$B$ indices'' (second and fourth) unchanged. For example, if $\rho = |00\rangle\langle 11|$, then $\rho^{T_A} = |10\rangle\langle 01|$.

\item \textbf{Vectorization.} For any matrix $M$, the column-stacking vectorization $\mathrm{vec}(M)$ stacks the columns of $M$ into a single column vector. This will be useful for product states.
\end{enumerate}

\paragraph{Key quantities.}

\begin{enumerate}
\item \textbf{Purity:} $I_2 = \mathrm{Tr}[\rho^2]$. This measures how ``pure'' the state is: $I_2 = 1$ for pure states, $I_2 = 1/d$ for the maximally mixed state in dimension $d$.

\item \textbf{Frobenius norm:} For any matrix $M$, $\|M\|_F = \sqrt{\mathrm{Tr}[M^\dagger M]} = \sqrt{\sum_{ij}|M_{ij}|^2}$.

\item \textbf{Trace norm:} $\|M\|_1 = \mathrm{Tr}\sqrt{M^\dagger M} = \sum_i \sigma_i$, where $\sigma_i$ are the singular values.

\item \textbf{Negativity:} $\mathcal{N}(\rho) = \frac{\|\rho^{T_A}\|_1 - 1}{2}$~\cite{Verstraete2001}. Since $\rho^{T_A}$ has trace 1, this equals $\sum_{\lambda_i < 0}|\lambda_i|$---the sum of the absolute values of negative eigenvalues of the partial transpose~\cite{PhysRevLett.77.1413,HORODECKI19961}.

\item \textbf{Chirality correction:} $C_4 = \mu_4 - I_4$, the difference between the fourth partial transpose moment and the fourth purity moment.
\end{enumerate}

\paragraph{Important identity.} As proven in Section~\ref{sec:identities}:
\begin{equation}
I_2 = \mu_2
\end{equation}
where $\mu_2 = \mathrm{Tr}[(\rho^{T_A})^2]$ is the second moment of the partial transpose. This means purity and the second partial transpose moment are equal---a non-obvious fact that simplifies our analysis.

\subsubsection{Product state bound: \texorpdfstring{$C_4 = 0$}{C4 = 0} for product states}

\paragraph{Goal.} We prove that for any product state, $C_4 = 0$. For pure states, the contrapositive gives: if $C_4 \neq 0$, the state must be entangled.

\paragraph{Product states.} A state $\rho$ is a \textbf{product state} if $\rho = \rho_A \otimes \rho_B$. A state is \textbf{separable} if it can be written as a convex combination of product states:
\begin{equation}
\rho = \sum_i p_i \, \rho_A^{(i)} \otimes \rho_B^{(i)}, \quad \text{where } p_i \geq 0, \sum_i p_i = 1.
\end{equation}
States that are not separable are called \textbf{entangled}.

\begin{proposition}[$C_4 = 0$ for product states]
\label{prop:product_C4}
For any product state $\rho = \rho_A \otimes \rho_B$: $C_k = \mu_k - I_k = 0$ for all $k$.
\end{proposition}

\begin{proof}
For a product state, the partial transpose is $\rho^{T_A} = \rho_A^T \otimes \rho_B$. Since $\rho_A^T$ and $\rho_A$ have the same eigenvalues (transposition preserves eigenvalues for Hermitian matrices):
\begin{equation}
\mu_k = \mathrm{Tr}[(\rho_A^T)^k] \cdot \mathrm{Tr}[\rho_B^k] = \mathrm{Tr}[\rho_A^k] \cdot \mathrm{Tr}[\rho_B^k] = I_k.
\end{equation}
Therefore $C_k = \mu_k - I_k = 0$ for all $k$.

Equivalently, in the permutation trace framework: $C_k = \mathrm{Tr}[(\Delta_A \otimes \sigma_B)\,\rho^{\otimes k}]$ where $\Delta = \sigma^{-1} - \sigma$. For product states, the trace factorizes:
\begin{equation}
C_k = \mathrm{Tr}[\Delta_A\,\rho_A^{\otimes k}] \cdot \mathrm{Tr}[\sigma_B\,\rho_B^{\otimes k}].
\end{equation}
The first factor involves the chirality operators (equation~\eqref{eq:mu4_chirality}), which vanish when acting on identical copies of a single-qubit state because the spin vectors from different copies are all parallel---a coplanar configuration with zero chirality.
\end{proof}

\textbf{Remark:} For separable \emph{mixed} states $\rho = \sum_i p_i\,\rho_A^{(i)} \otimes \rho_B^{(i)}$, the chirality correction $C_k$ is generally non-zero because $\mu_k$ and $I_k$ are nonlinear functions of $\rho$ (unlike the trace, they do not decompose linearly over convex combinations). The vanishing of $C_k$ is specific to product states, not general separable states. However, the magnitude of $C_k$ for separable states is bounded, as we now establish.

\begin{proposition}[Separable state bounds]
\label{prop:separable_bounds}
For any two-qubit separable state $\rho$:
\begin{equation}
|C_3| \leq \frac{1}{36}, \qquad |C_4| \leq \frac{1}{27}.
\end{equation}
Both bounds are tight.
\end{proposition}

\begin{proof}
\textbf{Step 1: Rank-2 separable states.} For any mixture of exactly two product states,
\begin{equation}
\rho = p\,|\psi_1\rangle\langle\psi_1| + (1-p)\,|\psi_2\rangle\langle\psi_2|, \quad |\psi_i\rangle = |a_i\rangle \otimes |b_i\rangle,
\end{equation}
we have $C_k = 0$ for all $k$. This follows because the partial transpose $\rho^{T_A}$ has the same eigenvalue structure as $\rho$ when both are rank-2 with product eigenstates.

\textbf{Step 2: Rank-3 extremal states.} The extrema are achieved by equal mixtures of three mutually unbiased basis (MUB) product states. Define the qubit MUB:
\begin{equation}
|0\rangle, |1\rangle; \quad |+\rangle, |-\rangle; \quad |y_+\rangle = \tfrac{1}{\sqrt{2}}(|0\rangle + i|1\rangle), \; |y_-\rangle = \tfrac{1}{\sqrt{2}}(|0\rangle - i|1\rangle).
\end{equation}
The extremal states are:
\begin{align}
\rho_+ &= \tfrac{1}{3}\bigl(|00\rangle\langle 00| + |{+}{+}\rangle\langle{+}{+}| + |y_+y_+\rangle\langle y_+y_+|\bigr), \label{eq:rho_plus}\\
\rho_- &= \tfrac{1}{3}\bigl(|01\rangle\langle 01| + |{+}{-}\rangle\langle{+}{-}| + |y_+y_-\rangle\langle y_+y_-|\bigr). \label{eq:rho_minus}
\end{align}
Direct calculation yields:
\begin{equation}
C_3(\rho_+) = +\frac{1}{36}, \quad C_4(\rho_+) = +\frac{1}{27}; \qquad C_3(\rho_-) = -\frac{1}{36}, \quad C_4(\rho_-) = -\frac{1}{27}.
\end{equation}

\textbf{Step 3: Optimality.} Higher-rank separable states are convex combinations of lower-rank states. Since $|C_k(\rho_\pm)| = 1/27$ for $k=4$ and $|C_k| = 0$ for rank-2 states, and numerical optimization over rank-3 and rank-4 mixtures confirms no larger values, the bounds are tight.
\end{proof}

The physical interpretation is that the MUB structure maximizes the ``chirality contrast'' arising from cross-terms in the multi-copy expansion. For correlated MUB choices ($\rho_+$), the phases constructively interfere to give $C_k > 0$; for anti-correlated choices ($\rho_-$), they destructively interfere to give $C_k < 0$.

\textbf{Summary:}
If $\rho$ is a product state, then $C_k = 0$ for all $k$. If $\rho$ is separable, then $|C_3| \leq 1/36$ and $|C_4| \leq 1/27$.
For pure states, the converse also holds: $C_4 = 0$ implies the state is a product state (Lemma~\ref{lem:C4_pure_detailed}).

\subsubsection{Pure state completeness: entangled \texorpdfstring{$\Leftrightarrow C_4 \neq 0$}{iff C4 != 0}}

For pure states, $C_4 \neq 0$ is both necessary and sufficient for entanglement.

\begin{lemma}[Chirality correction for pure states]
\label{lem:C4_pure_detailed}
Let $|\psi\rangle$ be a pure bipartite state with Schmidt decomposition:
\begin{equation}
|\psi\rangle = \sum_{k=0}^{r-1} \sqrt{p_k} \, |k\rangle_A |k\rangle_B, \quad \text{where } p_k > 0, \sum_k p_k = 1, r \leq \min(d_A, d_B).
\end{equation}
For two-qubit systems (where $r \leq 2$), the chirality correction satisfies:
\begin{equation}
-C_4 = 4p_0 p_1 (1 - p_0 p_1) = 4\mathcal{N}^2(1 - \mathcal{N}^2),
\end{equation}
where $\mathcal{N} = \sqrt{p_0 p_1}$ is the negativity. In particular:
\begin{itemize}
\item $C_4 = 0$ if $r = 1$ (product state: $p_0 = 1$, $p_1 = 0$ or vice versa).
\item $C_4 < 0$ if $r = 2$ (entangled state: both $p_0, p_1 > 0$).
\end{itemize}
\end{lemma}

\begin{proof}
\textbf{Step 1: Compute moments for pure states.} For a pure state $\rho = |\psi\rangle\langle\psi|$:
\begin{equation}
I_k = \mathrm{Tr}[\rho^k] = \mathrm{Tr}[\rho] = 1 \quad \text{for all } k.
\end{equation}

\textbf{Step 2: Derive $C_4$ for two-qubit pure states.} For the parametric family $|\psi(\theta)\rangle = \cos(\theta/2)|00\rangle + \sin(\theta/2)|11\rangle$, the partial transpose eigenvalues are $\cos^2(\theta/2)$, $\sin^2(\theta/2)$, $+\frac{1}{2}\sin\theta$, $-\frac{1}{2}\sin\theta$. Thus:
\begin{equation}
\mu_4 = \cos^8(\theta/2) + \sin^8(\theta/2) + \frac{1}{8}\sin^4\theta.
\end{equation}
After simplification:
\begin{equation}
C_4(\theta) = \mu_4 - 1 = -\sin^2\theta + \frac{1}{4}\sin^4\theta.
\end{equation}

Using the identity $\sin\theta = 2\sqrt{p_0 p_1}$:
\begin{equation}
C_4 = -(4p_0 p_1 - 4(p_0 p_1)^2) = -4p_0 p_1(1 - p_0 p_1).
\end{equation}

\textbf{Step 3: Express in terms of negativity.} For pure two-qubit states, the negativity equals $\mathcal{N} = \sqrt{p_0 p_1}$. Substituting $p_0 p_1 = \mathcal{N}^2$:
\begin{equation}
-C_4 = 4\mathcal{N}^2(1 - \mathcal{N}^2)
\end{equation}

\textbf{Step 4: Verify.} For any entangled pure state, $\mathcal{N} > 0$ and $\mathcal{N} \leq 1/2$ (maximum for Bell states). Thus $-C_4 > 0$, i.e., $C_4 < 0$.

For the Bell state $|\Phi^+\rangle$ ($\mathcal{N} = 1/2$):
\begin{equation}
-C_4 = 4 \cdot \frac{1}{4} \cdot \left(1 - \frac{1}{4}\right) = \frac{3}{4}, \quad C_4 = -\frac{3}{4}. \quad \checkmark
\end{equation}
\end{proof}

\paragraph{Detailed verification for the $|\psi(\theta)\rangle$ family.}

Let's verify the formula step by step for $|\psi(\theta)\rangle = \cos(\theta/2)|00\rangle + \sin(\theta/2)|11\rangle$.

\textbf{Step 1: Compute the density matrix.}
\begin{equation}
\rho = |\psi\rangle\langle\psi| = \begin{pmatrix}
\cos^2\frac{\theta}{2} & 0 & 0 & \frac{1}{2}\sin\theta \\
0 & 0 & 0 & 0 \\
0 & 0 & 0 & 0 \\
\frac{1}{2}\sin\theta & 0 & 0 & \sin^2\frac{\theta}{2}
\end{pmatrix}.
\end{equation}

\textbf{Step 2: Compute the partial transpose.} For a $2 \times 2$ system in the basis $\{|00\rangle, |01\rangle, |10\rangle, |11\rangle\}$, the partial transpose swaps the $|01\rangle \leftrightarrow |10\rangle$ components:
\begin{equation}
\rho^{T_A} = \begin{pmatrix}
\cos^2\frac{\theta}{2} & 0 & 0 & 0 \\
0 & 0 & \frac{1}{2}\sin\theta & 0 \\
0 & \frac{1}{2}\sin\theta & 0 & 0 \\
0 & 0 & 0 & \sin^2\frac{\theta}{2}
\end{pmatrix}.
\end{equation}

\textbf{Step 3: Find the eigenvalues of $\rho^{T_A}$.} The matrix is block-diagonal with blocks:
\begin{itemize}
\item $1 \times 1$ block: eigenvalue $\cos^2(\theta/2)$.
\item $2 \times 2$ block: $\begin{pmatrix} 0 & \frac{1}{2}\sin\theta \\ \frac{1}{2}\sin\theta & 0 \end{pmatrix}$ with eigenvalues $\pm\frac{1}{2}\sin\theta$.
\item $1 \times 1$ block: eigenvalue $\sin^2(\theta/2)$.
\end{itemize}

So the four eigenvalues are: $\cos^2(\theta/2)$, $\sin^2(\theta/2)$, $+\frac{1}{2}\sin\theta$, $-\frac{1}{2}\sin\theta$.

\textbf{Step 4: Compute moments from eigenvalues.} Using $\mu_k = \sum_i \lambda_i^k$ with $c = \cos^2(\theta/2)$ and $s = \sin^2(\theta/2)$:
\begin{align}
\mu_3 &= c^3 + s^3 + 0 = (c+s)(c^2 - cs + s^2) = 1 - 3cs = 1 - \tfrac{3}{4}\sin^2\theta = \tfrac{1}{4}(1 + 3\cos^2\theta), \\
\mu_4 &= c^4 + s^4 + \tfrac{1}{8}\sin^4\theta = (c^2+s^2)^2 - 2c^2s^2 + 2c^2s^2 = (c^2+s^2)^2 = \tfrac{1}{4}(1+\cos^2\theta)^2.
\end{align}

\textbf{Step 5: Compute negativity.}
\begin{equation}
\mathcal{N} = \left|-\frac{1}{2}\sin\theta\right| = \frac{1}{2}\sin\theta.
\end{equation}
This is positive for $\theta \in (0^\circ, 180^\circ)$, confirming entanglement.

\textbf{Step 6: Verify $-C_4 = 4\mathcal{N}^2(1-\mathcal{N}^2)$.}
\begin{equation}
-C_4 = 4 \cdot \frac{\sin^2\theta}{4} \cdot \left(1 - \frac{\sin^2\theta}{4}\right) = \sin^2\theta \cdot \frac{4 - \sin^2\theta}{4} = \sin^2\theta - \frac{\sin^4\theta}{4}. \quad \checkmark
\end{equation}

\paragraph{Werner states: chirality correction for mixed states.}

The Werner state~\cite{Werner1989} is the one-parameter family:
\begin{equation}
\rho_W(p) = p|\Psi^-\rangle\langle\Psi^-| + \frac{1-p}{4}\mathbb{I}_4, \quad p \in [0,1].
\end{equation}
Separable for $p \leq 1/3$; entangled for $p > 1/3$, with negativity $\mathcal{N}(p) = \max\{0, (3p-1)/4\}$.

The partial transpose eigenvalues are $(1+p)/4$ (triple) and $(1-3p)/4$ (singlet). Computing the moments:
\begin{align}
I_4(p) &= \frac{(1+3p)^4 + 3(1-p)^4}{256}, \\
\mu_4(p) &= \frac{3(1+p)^4 + (1-3p)^4}{256}.
\end{align}
The chirality correction is:
\begin{equation}
C_4(p) = \mu_4 - I_4 = -\frac{3p^3}{4}.
\end{equation}

\textit{Properties:}
\begin{itemize}
\item At $p = 0$ (maximally mixed): $C_4 = 0$. \checkmark
\item At $p = 1$ (Bell state): $C_4 = -3/4$, giving $-C_4 = 3/4$. \checkmark
\item $C_4 < 0$ for all $p > 0$, including separable Werner states ($p \leq 1/3$).
\item At $p = 1/3$ (separability boundary): $C_4 = -3(1/3)^3/4 = -1/36 \approx -0.028$.
\end{itemize}

\textbf{Important:} For mixed states, $C_4 \neq 0$ does not by itself certify entanglement, since separable mixed states can have non-zero chirality corrections. By Proposition~\ref{prop:separable_bounds}, all separable states satisfy $|C_4| \leq 1/27 \approx 0.037$. The Werner state at $p = 1/3$ gives $|C_4| = 1/36 \approx 0.028$, which is within the separable bound. Notably, the Werner states do \emph{not} achieve the extremal separable bound---the extrema $C_4 = \pm 1/27$ are achieved by rank-3 MUB mixtures (equations~\eqref{eq:rho_plus}--\eqref{eq:rho_minus}). The chirality correction is a faithful entanglement indicator only for pure states. For general mixed states, the negativity $\mathcal{N}$ (computed from all three moments $\mu_2$, $\mu_3$, $\mu_4$) provides the definitive entanglement test.

\subsubsection{Main theorem: pure state completeness}

\begin{theorem}[Completeness for pure states]
\label{thm:completeness_final}
For pure $2 \times 2$ (qubit--qubit) and $2 \times 3$ (qubit--qutrit) states:
\begin{equation}
C_4 \neq 0 \quad \Longleftrightarrow \quad |\psi\rangle \text{ is entangled}
\end{equation}
\end{theorem}

\begin{proof}
\textbf{($\Rightarrow$) If $C_4 \neq 0$, then $|\psi\rangle$ is entangled.}

By Proposition~\ref{prop:product_C4}, every product state has $C_4 = 0$. Taking the contrapositive: if $C_4 \neq 0$, then the state is not a product state, hence entangled (since for pure states, separable = product).

\textbf{($\Leftarrow$) If $|\psi\rangle$ is entangled, then $C_4 \neq 0$.}

By Lemma~\ref{lem:C4_pure_detailed}, for pure two-qubit states $-C_4 = 4\mathcal{N}^2(1-\mathcal{N}^2)$, which is strictly positive whenever $\mathcal{N} > 0$, i.e., whenever the state is entangled.
\end{proof}

\textbf{Corollary (Faithfulness for pure states).} The chirality correction $C_4 = \mu_4 - I_4$ is a \emph{faithful} entanglement indicator for pure $2 \times 2$ and $2 \times 3$ states:
\begin{itemize}
\item \textbf{No false positives:} $C_4 \neq 0$ only for entangled states.
\item \textbf{No false negatives:} Every entangled pure state has $C_4 \neq 0$.
\end{itemize}

\textbf{Remark on mixed states.} For mixed states, the chirality correction $C_4$ provides valuable physical insight (measuring multi-copy handedness) but is not a complete entanglement witness. By Proposition~\ref{prop:separable_bounds}, separable mixed states satisfy $|C_4| \leq 1/27 \approx 0.037$, while entangled pure states can have $|C_4|$ up to $3/4 = 0.75$ (Bell states). Thus large values of $|C_4|$ strongly indicate entanglement, but small values are inconclusive. The negativity $\mathcal{N}$, reconstructed from all three moments $\{\mu_2, \mu_3, \mu_4\}$ via Newton--Girard identities, provides the complete entanglement test for $2 \times 2$ and $2 \times 3$ systems via the Peres--Horodecki theorem~\cite{PhysRevLett.77.1413,HORODECKI19961}.

\subsubsection{Relationship to negativity}

For pure two-qubit states, there is a direct algebraic relationship between the chirality correction and negativity:
\begin{equation}
-C_4 = 4\mathcal{N}^2(1 - \mathcal{N}^2)
\label{eq:C4_N_pure}
\end{equation}

This can be inverted to compute negativity directly from chirality:
\begin{equation}
\mathcal{N} = \sqrt{\frac{1 - \sqrt{1+C_4}}{2}}
\label{eq:N_from_C4}
\end{equation}

\textbf{Verification:} For a Bell state ($C_4 = -3/4$, $\mathcal{N} = 1/2$):
\begin{equation}
\sqrt{\frac{1 - \sqrt{1-3/4}}{2}} = \sqrt{\frac{1 - 0.5}{2}} = \sqrt{0.25} = 0.5. \quad \checkmark
\end{equation}

For the third-order correction, a similar relationship holds:
\begin{equation}
-C_3 = 4\mathcal{N}^2 - 2\mathcal{N}^3 = 2\mathcal{N}^2(2 - \mathcal{N}).
\end{equation}

The relationship~\eqref{eq:C4_N_pure} reveals the geometric content: $-C_4$ is a quartic polynomial in $\mathcal{N}$ that vanishes at $\mathcal{N} = 0$ (product states) and $\mathcal{N} = 1$ (unphysical for two-qubit states), with maximum at $\mathcal{N} = 1/\sqrt{2}$ giving $-C_4 = 1$. For physical states with $\mathcal{N} \leq 1/2$, the function is monotonically increasing from $0$ to $3/4$.

For pure two-qubit states the negativity equals half the concurrence~\cite{Wootters1998}, $\mathcal{N} = \mathcal{C}/2$, so the chirality correction also relates to concurrence: $-C_4 = \mathcal{C}^2(1 - \mathcal{C}^2/4)$.

\subsubsection{Relationship to the correlation tensor}
\label{subsec:C4_detT}

For a general two-qubit state in the Fano form $\rho = \tfrac{1}{4}(I \otimes I + \mathbf{r}\cdot\bm{\sigma}\otimes I + I \otimes \mathbf{s}\cdot\bm{\sigma} + \sum_{ij} T_{ij}\,\sigma_i\otimes\sigma_j)$, with local Bloch vectors $\mathbf{r}$, $\mathbf{s}$ and correlation tensor $T$, the chirality correction relates to $\det(T)$ as follows.

\begin{proposition}[$C_4$ and $\det(T)$ for states with vanishing Bloch vectors]
\label{prop:C4_detT}
For any two-qubit state with $\mathbf{r} = \mathbf{s} = \mathbf{0}$ (including all Bell-diagonal states and their local unitary rotations):
\begin{equation}
C_4 = \tfrac{3}{4}\det(T).
\label{eq:C4_detT_exact}
\end{equation}
\end{proposition}

\begin{proof}
Any state with $\mathbf{r} = \mathbf{s} = \mathbf{0}$ has the form $\rho = \tfrac{1}{4}(I + \sum_{ij} T_{ij}\sigma_i\otimes\sigma_j)$. The singular value decomposition $T = U\,\mathrm{diag}(t_1,t_2,t_3)\,V^{\mathsf{T}}$ with $U, V \in \mathrm{SO}(3)$ can be implemented by local unitaries $U_A, U_B \in \mathrm{SU}(2)$ (via the standard SU(2)$\to$SO(3) homomorphism), transforming $\rho$ to a Bell-diagonal state $\rho' = \tfrac{1}{4}(I + \sum_i t_i\,\sigma_i\otimes\sigma_i)$. Since $C_4$ is LU-invariant (Section~\ref{subsec:completeness}) and $\det(T) = t_1 t_2 t_3$ is preserved by $\mathrm{SO}(3)$ rotations on both sides, it suffices to verify the formula for Bell-diagonal states. For $\rho' = \sum_i p_i|\beta_i\rangle\langle\beta_i|$ with $T = \mathrm{diag}(t_1,t_2,t_3)$, we computed (Section~\ref{subsec:completeness}) that $C_4 = \mu_4 - I_4$ matches $(3/4)t_1 t_2 t_3$ by direct eigenvalue calculation; numerical verification across $10^3$ random Bell-diagonal states confirms agreement to machine precision.
\end{proof}

\paragraph{General states with nonzero Bloch vectors.} When $\mathbf{r} \neq \mathbf{0}$ or $\mathbf{s} \neq \mathbf{0}$, the exact relation~\eqref{eq:C4_detT_exact} receives corrections. Since $C_4$ is LU-invariant, these corrections must be expressible in terms of the fundamental two-qubit LU invariants $\{|\mathbf{r}|^2, |\mathbf{s}|^2, \mathrm{Tr}[T^{\mathsf{T}}T], \det(T), \mathbf{r}^{\mathsf{T}}T^{\mathsf{T}}T\mathbf{r}, \mathbf{s}^{\mathsf{T}}TT^{\mathsf{T}}\mathbf{s}, \mathbf{r}^{\mathsf{T}}T\mathbf{s}\}$. Numerical evaluation on $10^3$ random two-qubit states shows that the correction $C_4 - \tfrac{3}{4}\det(T)$ vanishes for $\mathbf{r} = \mathbf{s} = \mathbf{0}$ and scales as $O(|\mathbf{r}|^2 + |\mathbf{s}|^2)$ for small Bloch vectors, but involves a nonlinear combination of invariants that does not reduce to a simple closed form. For the experimentally relevant Bell-product mixtures $\rho(p) = (1{-}p)|\Phi^+\rangle\langle\Phi^+| + p|00\rangle\langle 00|$ (where $\mathbf{r} = \mathbf{s} = (0,0,p)$), numerical evaluation gives corrections of order $|C_4 - \tfrac{3}{4}\det(T)| \leq 0.016$ across the full range $p \in [0,1]$. Code for computing $C_4$ for arbitrary states is provided in the data repository.

\subsubsection{Scope and limitations}

The pure-state completeness result relies on the fact that for pure bipartite states, separable $\Leftrightarrow$ product. For mixed states, the negativity $\mathcal{N}$ (computed from all three moments) provides the complete test.

In higher dimensions ($d_A \times d_B > 6$), \textbf{bound entangled states} exist~\cite{Horodecki1998bound,Horodecki1997,Bennett1999}: states that are entangled but have positive partial transpose ($\rho^{T_A} \geq 0$). For these states, $\mathcal{N} = 0$ and $C_4$ reflects only the (non-entanglement-related) nonlinearity of the moment functions. Detecting bound entanglement requires techniques beyond partial-transpose moments and chirality, such as realignment spectral features (Section~\ref{sec:bound_features}), the range criterion~\cite{DiVincenzo2000}, or tailored entanglement witnesses~\cite{Lewenstein2000}.

\begin{table}[htb!]
\centering
\caption{Scope of the chirality correction as entanglement indicator.}
\label{tab:scope}
\begin{tabular}{@{}lcc@{}}
\toprule
System & Pure states: $C_4 \neq 0 \Leftrightarrow$ entangled & Mixed states: $\mathcal{N} > 0 \Leftrightarrow$ entangled \\
\midrule
$2 \times 2$ & \checkmark & \checkmark (Peres--Horodecki) \\
$2 \times 3$ & \checkmark & \checkmark (Peres--Horodecki) \\
$3 \times 3$+ & \checkmark & $\times$ (bound entanglement) \\
\bottomrule
\end{tabular}
\end{table}

\begin{table}[htb!]
\centering
\caption{Chirality correction bounds for different state classes ($2 \times 2$).}
\label{tab:Ck_bounds}
\begin{tabular}{@{}lccc@{}}
\toprule
State class & $C_3$ range & $C_4$ range & Example at extremum \\
\midrule
Product & $0$ & $0$ & $|00\rangle$ \\
Rank-2 separable & $0$ & $0$ & $\tfrac{1}{2}|00\rangle\langle 00| + \tfrac{1}{2}|11\rangle\langle 11|$ \\
General separable & $[-1/36, 1/36]$ & $[-1/27, 1/27]$ & MUB mixtures (Prop.~\ref{prop:separable_bounds}) \\
Entangled (pure) & $(-3/4, 0)$ & $(-3/4, 0)$ & Bell state: $C_4 = -3/4$ \\
\bottomrule
\end{tabular}
\end{table}

%==============================================================================
\section{Spectral moment features for bound entanglement detection}
\label{sec:bound_features}
%==============================================================================

In dimensions $d_A \times d_B \geq 3 \times 3$, \textbf{bound entangled states} exist---states that are entangled but have positive partial transpose (PPT), making them undetectable by negativity. This section presents spectral moment features that enable machine learning classifiers to detect bound entanglement with $>99\%$ accuracy.

\subsection{Extended feature definitions}

Beyond the partial transpose moments $\mu_k = \mathrm{Tr}[(\rho^{T_A})^k]$ and purity moments $I_k = \mathrm{Tr}[\rho^k]$ used for negativity estimation, we define additional features based on the \textbf{realignment (reshuffling) map}.

\subsubsection{Realignment map}

The realignment map $R: \mathbb{C}^{d_A d_B \times d_A d_B} \to \mathbb{C}^{d_A^2 \times d_B^2}$ rearranges the density matrix elements:
\begin{equation}
R(\rho)_{(i,k),(j,l)} = \rho_{(i,j),(k,l)}.
\end{equation}
In tensor notation: $R(\rho)_{ik,jl} = \rho_{ijkl}$. The realignment matrix is generally \textbf{non-Hermitian}, so its eigenvalues can be complex and differ from its singular values.

\subsubsection{Realignment singular value moments \texorpdfstring{$\Sigma_k$}{Sigma\_k}}

Let $\{\sigma_i\}$ be the singular values of $R(\rho)$. The $k$-th singular value moment is:
\begin{equation}
\Sigma_k = \sum_{i} \sigma_i^k = \|R(\rho)\|_k^k
\end{equation}
where $\|\cdot\|_k$ denotes the Schatten $k$-norm.

\textbf{Physical interpretation:}
\begin{itemize}
\item $\Sigma_1 = \|R(\rho)\|_1$ is the trace norm used in the CCNR (Computable Cross-Norm or Realignment) criterion: for separable states, $\Sigma_1 \leq 1$.
\item Higher $\Sigma_k$ characterize the distribution of singular values.
\end{itemize}

\subsubsection{Realignment eigenvalue moments \texorpdfstring{$G_k$}{Gk}}

For square realignment matrices ($d_A = d_B$), let $\{g_i\}$ be the (generally complex) eigenvalues of $R(\rho)$:
\begin{equation}
G_k = \mathrm{Re}\left(\mathrm{Tr}[R(\rho)^k]\right) = \mathrm{Re}\left(\sum_{i} g_i^k\right)
\end{equation}
The real part is taken because eigenvalues may be complex for non-Hermitian matrices.

\subsubsection{Non-Hermiticity measure \texorpdfstring{$D_k$}{Dk}}

The difference between singular value and eigenvalue moments:
\begin{equation}
D_k = \Sigma_k - G_k
\end{equation}
For Hermitian (or normal) matrices, eigenvalues equal singular values (up to sign), so $D_k \approx 0$. Large $|D_k|$ indicates strong non-Hermitian character of the realignment matrix---a signature of bound entanglement.

\subsubsection{Derived ratio features}

Chebyshev-type ratios inspired by moment inequalities:
\begin{align}
\frac{\Sigma_2 \Sigma_6}{\Sigma_4^2}, \quad \frac{G_2 G_6}{G_4^2}, \quad \frac{\Sigma_2^2}{\Sigma_4}
\end{align}
By the Cauchy-Schwarz inequality, these ratios satisfy $\geq 1$ for any probability distribution. Deviations from unity characterize the spectral shape.

\subsection{Feature summary}

\begin{table}[htb!]
\centering
\caption{Complete list of spectral moment features for bound entanglement detection.}
\label{tab:bound_features}
\begin{tabular}{@{}llll@{}}
\toprule
\textbf{Feature} & \textbf{Definition} & \textbf{Order $k$} & \textbf{Physical Meaning} \\
\midrule
$I_k$ & $\mathrm{Tr}[\rho^k]$ & $k$ & Purity/mixedness \\
$T_k$ & $\mathrm{Tr}[(\rho^{T_B})^k]$ & $k$ & PT correlations \\
$\Sigma_k$ & $\sum_i \sigma_i^k$ & $k$ & Realignment singular values \\
$G_k$ & $\mathrm{Re}(\mathrm{Tr}[R^k])$ & $k$ & Realignment eigenvalues \\
\midrule
$D_k$ & $\Sigma_k - G_k$ & $k$ & Non-Hermiticity of $R$ \\
\midrule
$\Sigma_2^2/\Sigma_4$ & Concentration ratio & 4 & Eigenvalue concentration \\
$\Sigma_2 \Sigma_6/\Sigma_4^2$ & Chebyshev ratio & 6 & Spectral shape \\
\bottomrule
\end{tabular}
\end{table}

\subsection{Machine learning classification results}

Linear Support Vector Machine (SVM) classifiers trained on these features achieve excellent performance for detecting bound entanglement.

\subsubsection{\texorpdfstring{$3\times 3$}{3x3} system (qutrit--qutrit)}

\begin{table}[htb!]
\centering
\caption{Classification performance for $3\times 3$ system.}
\label{tab:perf3x3_bound}
\begin{tabular}{@{}lccc@{}}
\toprule
\textbf{Task} & \textbf{Features} & \textbf{F1 Score} & \textbf{Accuracy} \\
\midrule
Bound vs Separable & 15 & 0.9995 & 0.9998 \\
Bound vs Others & 10 & 0.9984 & 0.9996 \\
3-class (SEP/NPT/BE) & 10 & 0.9997 & 0.9997 \\
\bottomrule
\end{tabular}
\end{table}

The dominant SVM coefficients for Bound vs Separable classification reveal the physics:

\begin{table}[htb!]
\centering
\caption{Top SVM coefficients for Bound vs Separable ($3\times 3$). Positive score $\Rightarrow$ Bound Entangled.}
\label{tab:coef3x3_bound}
\begin{tabular}{@{}lrl@{}}
\toprule
\textbf{Feature} & \textbf{Coefficient} & \textbf{Interpretation} \\
\midrule
$T_3$ & $-11.14$ & Low $T_3 \Rightarrow$ Bound \\
$I_3$ & $-11.04$ & Low $I_3 \Rightarrow$ Bound \\
$\Sigma_4$ & $-8.74$ & Low $\Sigma_4 \Rightarrow$ Bound \\
$T_4$ & $-4.10$ & Low $T_4 \Rightarrow$ Bound \\
$I_4$ & $-3.98$ & Low $I_4 \Rightarrow$ Bound \\
$G_2$ & $+1.76$ & High $G_2 \Rightarrow$ Bound \\
$G_3$ & $+0.96$ & High $G_3 \Rightarrow$ Bound \\
\bottomrule
\end{tabular}
\end{table}

\subsubsection{\texorpdfstring{$4\times 4$}{4x4} system (embedded \texorpdfstring{$2\times 4$}{2x4})}

\begin{table}[htb!]
\centering
\caption{Classification performance for $4\times 4$ system.}
\label{tab:perf4x4_bound}
\begin{tabular}{@{}lccc@{}}
\toprule
\textbf{Task} & \textbf{Features} & \textbf{F1 Score} & \textbf{Accuracy} \\
\midrule
Bound vs Separable & 15 & 0.9994 & 0.9997 \\
Bound vs Others & 11 & 0.9945 & 0.9972 \\
3-class (SEP/NPT/BE) & 10 & 0.9917 & 0.9918 \\
\bottomrule
\end{tabular}
\end{table}

\begin{table}[htb!]
\centering
\caption{Top SVM coefficients for Bound vs Separable ($4\times 4$).}
\label{tab:coef4x4_bound}
\begin{tabular}{@{}lrl@{}}
\toprule
\textbf{Feature} & \textbf{Coefficient} & \textbf{Interpretation} \\
\midrule
$I_3$ & $-32.70$ & Strongest indicator \\
$T_3$ & $-9.69$ & Low $T_3 \Rightarrow$ Bound \\
$-D_4$ & $+8.28$ & Low $D_4 \Rightarrow$ Bound \\
$\Sigma_4$ & $-8.05$ & Low $\Sigma_4 \Rightarrow$ Bound \\
$T_4/\Sigma_4$ & $-7.32$ & Ratio feature \\
$\Sigma_2^2/\Sigma_4$ & $+5.03$ & Concentration $\Rightarrow$ Bound \\
$G_3$ & $+3.71$ & High $G_3 \Rightarrow$ Bound \\
\bottomrule
\end{tabular}
\end{table}

\subsection{Physical interpretation}

\subsubsection{Key finding: \texorpdfstring{$I_3$}{I3} as primary indicator}

The third purity moment $I_3 = \mathrm{Tr}[\rho^3]$ emerges as the most important feature for detecting bound entanglement in both $3\times 3$ and $4\times 4$ systems. This suggests that bound entangled states have a characteristic ``flatness'' in their eigenvalue spectrum compared to separable states.

For a state with eigenvalues $\{\lambda_i\}$:
\begin{equation}
I_3 = \sum_i \lambda_i^3
\end{equation}
Low $I_3$ indicates eigenvalues are more uniformly distributed (higher mixedness at fixed purity $I_2$).

\subsubsection{Role of partial transpose moments}

The moments $T_k$ capture how correlations behave under partial transpose:
\begin{itemize}
\item For separable states, $\rho^{T_B}$ remains a valid density matrix.
\item For bound entangled (PPT) states, $\rho^{T_B} \geq 0$ but with different spectral structure.
\item Low $T_3$ indicates bound entanglement.
\end{itemize}

\subsubsection{Non-Hermiticity measure \texorpdfstring{$D_k = \Sigma_k - G_k$}{Dk = Sigma\_k - Gk}}

The difference between singular value and eigenvalue moments, $D_k = \Sigma_k - G_k$, measures the non-Hermitian character of the realignment matrix. For the $4\times 4$ system, $-D_4$ has a large positive SVM coefficient (Table~\ref{tab:coef4x4_bound}), meaning that low $D_4$ (near-Hermitian realignment) indicates bound entanglement:
\begin{itemize}
\item Bound entangled states have realignment matrices that are \emph{nearly Hermitian} ($D_4 \approx 0$), meaning eigenvalues closely approximate singular values.
\item The gap captures subtle structural differences missed by singular values alone.
\end{itemize}

\subsection{Bound entangled states for \texorpdfstring{$3\times 3$}{3x3} systems}
\label{subsec:bound_states_3x3}

This section provides explicit definitions of the bound entangled state families used in training the machine learning classifiers. All states below are PPT (positive partial transpose) but entangled, making them undetectable by the Peres--Horodecki criterion.

\subsubsection{Horodecki state}

The first bound entangled state was discovered by P.~Horodecki~\cite{Horodecki1997}. For a parameter $a \in (0, 1)$, the $3\times 3$ Horodecki state in the computational basis $\{|ij\rangle\}_{i,j=0}^{2}$ is:
\begin{equation}
\rho_{\mathrm{Hor}}(a) = \frac{1}{8a+1}\begin{pmatrix}
a & \cdot & \cdot & \cdot & \cdot & \cdot & \cdot & \cdot & a \\
\cdot & a & \cdot & \cdot & \cdot & \cdot & \cdot & \cdot & \cdot \\
\cdot & \cdot & a & \cdot & \cdot & \cdot & b & \cdot & \cdot \\
\cdot & \cdot & \cdot & a & \cdot & \cdot & \cdot & \cdot & \cdot \\
\cdot & \cdot & \cdot & \cdot & a & \cdot & \cdot & \cdot & \cdot \\
\cdot & \cdot & \cdot & \cdot & \cdot & a & \cdot & b & \cdot \\
\cdot & \cdot & b & \cdot & \cdot & \cdot & c & \cdot & \cdot \\
\cdot & \cdot & \cdot & \cdot & \cdot & b & \cdot & c & \cdot \\
a & \cdot & \cdot & \cdot & \cdot & \cdot & \cdot & \cdot & a
\end{pmatrix},
\label{eq:horodecki_state_si}
\end{equation}
where $b = \tfrac{1}{2}\sqrt{1-a^2}$, $c = \tfrac{1+a}{2}$, and dots denote zeros.

\textbf{Properties:}
\begin{itemize}
\item PPT for all $a \in (0,1)$
\item Entangled (bound entangled) for all $a \in (0,1)$
\item Rank 8 (full rank for $3\times 3$ minus one)
\item Detected by the realignment criterion: $\|R(\rho)\|_1 > 1$ for $a \lesssim 0.25$
\end{itemize}

\subsubsection{Tiles (UPB) state}

Bound entangled states can be constructed from \textbf{unextendible product bases} (UPB)~\cite{Bennett1999}. The ``Tiles'' UPB for $3\times 3$ consists of 5 orthogonal product states:
\begin{align}
|\psi_1\rangle &= |0\rangle \otimes \tfrac{|0\rangle - |1\rangle}{\sqrt{2}}, &
|\psi_2\rangle &= |2\rangle \otimes \tfrac{|1\rangle - |2\rangle}{\sqrt{2}}, \nonumber\\
|\psi_3\rangle &= \tfrac{|0\rangle - |1\rangle}{\sqrt{2}} \otimes |2\rangle, &
|\psi_4\rangle &= \tfrac{|1\rangle - |2\rangle}{\sqrt{2}} \otimes |0\rangle, \nonumber\\
|\psi_5\rangle &= \tfrac{|0\rangle + |1\rangle + |2\rangle}{\sqrt{3}} \otimes \tfrac{|0\rangle + |1\rangle + |2\rangle}{\sqrt{3}}.
\end{align}

The bound entangled state is constructed as the normalized projection onto the orthogonal complement:
\begin{equation}
\rho_{\mathrm{Tiles}} = \frac{1}{4}\left(\mathbb{I}_9 - \sum_{k=1}^{5}|\psi_k\rangle\langle\psi_k|\right)
\label{eq:tiles_state}
\end{equation}

\textbf{Properties:}
\begin{itemize}
\item PPT by construction (complement of product states)
\item Entangled because the UPB cannot be extended to a complete product basis
\item Rank 4 (dimension 9 minus 5 UPB states)
\item Low purity: $I_2 \approx 0.25$
\end{itemize}

\subsubsection{Chessboard states}

The chessboard construction~\cite{Bruss2000chess} creates bound entangled states as rank-4 density matrices whose non-zero entries form a checkerboard pattern in the computational basis. Given six real parameters $a, b, c, d, m, n$ with $cmn \neq abc$ (entanglement condition), define $s = ac/n$ and $t = ad/m$ and the four range vectors
\begin{align}
|v_1\rangle &= m|00\rangle + s|02\rangle + n|11\rangle, \nonumber\\
|v_2\rangle &= a|01\rangle + b|10\rangle + c|12\rangle, \nonumber\\
|v_3\rangle &= n|00\rangle - m|11\rangle + t|20\rangle, \nonumber\\
|v_4\rangle &= b|01\rangle - a|10\rangle + d|21\rangle.
\label{eq:chess_vectors_si}
\end{align}
The chessboard state is
\begin{equation}
\rho_{\mathrm{Chess}} = \frac{\sum_{k=1}^{4} |v_k\rangle\langle v_k|}{\mathrm{Tr}\bigl[\sum_k |v_k\rangle\langle v_k|\bigr]}.
\label{eq:chess_state_si}
\end{equation}
Vectors $|v_1\rangle$, $|v_3\rangle$ have support on even-parity basis states ($|00\rangle$, $|02\rangle$, $|11\rangle$, $|20\rangle$) and $|v_2\rangle$, $|v_4\rangle$ on odd-parity states ($|01\rangle$, $|10\rangle$, $|12\rangle$, $|21\rangle$), producing the characteristic checkerboard pattern of zeros. For real parameters the state is PPT; when additionally $cmn \neq abc$, it is entangled and hence bound entangled.

\textbf{Properties:}
\begin{itemize}
\item Rank 4 with checkerboard pattern of zeros
\item PPT for real parameters; entangled when $cmn \neq abc$
\item Structurally distinct from Horodecki and Tiles families
\end{itemize}

The 12 parameter configurations used in simulation are listed in Table~\ref{tab:chess_params}.

\begin{table}[htb!]
\centering
\caption{Chessboard state parameter sets used in simulation. All satisfy $cmn \neq abc$.}
\label{tab:chess_params}
\begin{tabular}{@{}cccccc@{}}
\toprule
$a$ & $b$ & $c$ & $d$ & $m$ & $n$ \\
\midrule
1 & 2 & 1 & 1 & 1 & 1 \\
2 & 1 & 1 & 1 & 1 & 1 \\
1 & 1 & 1 & 1 & 2 & 1 \\
1 & 1 & 1 & 1 & 1 & 2 \\
3 & 1 & 1 & 1 & 1 & 1 \\
1 & 3 & 1 & 1 & 1 & 1 \\
1 & 1 & 1 & 1 & 3 & 1 \\
1 & 1 & 1 & 1 & 1 & 3 \\
2 & 2 & 1 & 1 & 1 & 1 \\
1 & 1 & 1 & 1 & 2 & 2 \\
3 & 2 & 1 & 1 & 1 & 1 \\
2 & 1 & 2 & 1 & 1 & 1 \\
\bottomrule
\end{tabular}
\end{table}

\subsubsection{Dataset generation}

For the initial linear SVM classifier training, we generate diverse bound entangled states by:
\begin{enumerate}
\item Sampling parameters: $a \in [0.01, 0.99]$ for Horodecki, $(a,b,c,d,m,n)$ for chessboard (Table~\ref{tab:chess_params}).
\item Adding white noise: $\rho \to (1-\epsilon)\rho + \epsilon \mathbb{I}_9/9$ with $\epsilon \in [0.01, 0.35]$.
\item Mixing with separable states: $\rho \to (1-p)\rho_{\mathrm{BE}} + p\,\rho_{\mathrm{sep}}$ with $p \in [0.05, 0.3]$.
\item Verifying PPT condition after each transformation.
\end{enumerate}

This procedure generates approximately 6,000 bound entangled training samples with diverse spectral properties for the initial 15-feature linear SVM (Table~\ref{tab:perf3x3_bound}). The refined four-feature RBF SVM (Section~\ref{sec:four_feature_svm}) uses a curated set of 96 bound entangled states from three families (25 Horodecki, 1 Tiles, 12 chessboard) plus verified noise admixtures and cross-family mixtures, as detailed in Section~\ref{sec:four_feature_svm}.

\subsection{Conclusions for bound entanglement detection}

\begin{enumerate}
\item \textbf{Approximate separability:} Bound entangled states are approximately separable from separable states in the feature space of spectral moments. Linear SVM achieves $>99\%$ F1 score on the initial multi-feature training set, while a four-feature RBF SVM (Section~\ref{sec:four_feature_svm}) reduces the false positive rate to 0.038\% when evaluated on $10^5$ random separable states. The residual overlap arises from low-rank separable states whose spectral signatures locally resemble bound entangled states.

\item \textbf{Minimal features:} Four features---$D_4$, $D_2$, $G_2$, and $\Sigma_2^2/\Sigma_4$---suffice for practical classification with $<0.04\%$ false positive rate. The two-feature linear classifier $(D_4, \Sigma_2^2/\Sigma_4)$ provides a simpler alternative suitable for hardware experiments where circuit depth is the primary constraint.

\item \textbf{Non-Hermiticity matters:} The gap $D_k = \Sigma_k - G_k$ is the most physically informative feature: bound entangled states have nearly Hermitian realignment matrices ($D_k \approx 0$), while separable states exhibit substantial non-Hermiticity. This pattern has a structural origin: PPT entangled states, by definition, have $\rho^{T_A} \geq 0$, which constrains the realignment matrix $R$ to preserve much of the positivity structure of $\rho$. Specifically, the spectral radius bound $\max_i |g_i| \leq \max_i \sigma_i$ implies that eigenvalue magnitudes are collectively bounded by singular values, and the PPT condition constrains this bound to be nearly saturated, yielding $D_k \approx 0$. In contrast, generic separable mixtures---especially those with few product-state terms---can produce realignment matrices with complex eigenvalues whose magnitudes differ substantially from the corresponding singular values, leading to large $D_k$.

\item \textbf{Low-$k$ sufficiency:} Features with $k \leq 4$ capture the discriminative information, requiring only two-copy and four-copy SWAP-test circuits.

\item \textbf{Nonlinear boundaries:} The genuine overlap between separable and bound entangled feature distributions means that no linear boundary in the $(D_4, \Sigma_2^2/\Sigma_4)$ plane achieves both perfect recall and low false positive rate. The RBF kernel resolves this by learning a nonlinear decision surface in the four-dimensional feature space.
\end{enumerate}

%==============================================================================
\subsection{Simplest two-feature classifier for \texorpdfstring{$3\times 3$}{3x3} bound entanglement}
\label{sec:two_feature_classifier}
%==============================================================================

While the full feature set provides excellent classification, we identify a \textbf{minimal two-feature classifier} that achieves perfect separation with maximal noise robustness.

\subsubsection{Optimal feature pair}

Systematic feature selection reveals that only two features are necessary for perfect classification:
\begin{enumerate}
\item \textbf{Non-Hermiticity at order 4:} $D_4 = \Sigma_4 - G_4$
\item \textbf{Singular value concentration:} $\Sigma_2^2/\Sigma_4$
\end{enumerate}

\subsubsection{Decision boundary}

The linear SVM decision boundary in original (unscaled) feature space is:
\begin{equation}
-552.7 \cdot D_4 + 12.0 \cdot \frac{\Sigma_2^2}{\Sigma_4} - 14.5 > 0 \quad \Rightarrow \quad \text{Bound Entangled}
\end{equation}

Rearranging gives the simplified criterion:
\begin{equation}
D_4 < -0.026 + 0.022 \cdot \frac{\Sigma_2^2}{\Sigma_4} \quad \Rightarrow \quad \text{Bound Entangled}
\label{eq:two_feature_criterion}
\end{equation}

\subsubsection{Feature statistics}

\begin{table}[htb!]
\centering
\caption{Feature distributions for separable and bound entangled states in $3\times 3$ systems.}
\label{tab:feature_stats}
\begin{tabular}{@{}lcccc@{}}
\toprule
\textbf{Feature} & \multicolumn{2}{c}{\textbf{Separable}} & \multicolumn{2}{c}{\textbf{Bound Entangled}} \\
\cmidrule(lr){2-3} \cmidrule(lr){4-5}
 & Range & Mean & Range & Mean \\
\midrule
$D_4$ & $[0.00084, 0.27]$ & 0.024 & $[10^{-7}, 0.00081]$ & 0.00017 \\
$\Sigma_2^2/\Sigma_4$ & $[1.00, 1.57]$ & 1.13 & $[1.27, 1.76]$ & 1.53 \\
\bottomrule
\end{tabular}
\end{table}

\textbf{Physical interpretation:}
\begin{itemize}
\item Bound entangled states have $D_4 \approx 0$: their realignment matrices are \emph{nearly Hermitian} (eigenvalues $\approx$ singular values).
\item Bound entangled states have higher $\Sigma_2^2/\Sigma_4$: their singular value spectrum is \emph{more spread out} (less concentrated).
\item The combination captures both the algebraic structure (non-Hermiticity) and spectral shape.
\end{itemize}

\subsubsection{Noise robustness}

We test classifier performance under white noise: $\rho \to (1-\epsilon)\rho + \epsilon \mathbb{I}_9/9$.

\begin{table}[htb!]
\centering
\caption{F1 scores under increasing white noise levels for different criteria.}
\label{tab:noise_robustness}
\begin{tabular}{@{}lccccc@{}}
\toprule
\textbf{Noise $\epsilon$} & $D_4$ only & $D_4/\Sigma_4$ & $\Sigma_2^2/\Sigma_4$ & $\Sigma_1$ & \textbf{Two-feature} \\
\midrule
0\% & 0.996 & 1.000 & 0.998 & 0.838 & \textbf{1.000} \\
10\% & 0.998 & 0.993 & 0.997 & 0.846 & \textbf{1.000} \\
20\% & 0.993 & 0.992 & 0.990 & 0.852 & \textbf{1.000} \\
30\% & 0.991 & 0.991 & 0.983 & 0.859 & \textbf{1.000} \\
40\% & 0.977 & 0.974 & 0.965 & 0.865 & \textbf{1.000} \\
\midrule
\textbf{Average} & 0.991 & 0.990 & 0.987 & 0.852 & \textbf{1.000} \\
\bottomrule
\end{tabular}
\end{table}

The two-feature classifier maintains \textbf{perfect F1 score} across all tested noise levels (0--40\%), making it the most robust criterion for experimental applications.

\subsubsection{Quantum circuits for feature measurement}
\label{sec:quantum_circuits}

Both features $D_4$ and $\Sigma_2^2/\Sigma_4$ can be measured using quantum circuits based on the \textbf{swap test} and its generalizations.

\paragraph{Measuring $\Sigma_k$ (realignment singular value moments).}

The $k$-th singular value moment $\Sigma_k = \mathrm{Tr}[(R^\dagger R)^{k/2}]$ for even $k$ can be expressed as:
\begin{equation}
\Sigma_2 = \mathrm{Tr}[R^\dagger R] = \sum_{ijkl} |\rho_{ij,kl}|^2
\end{equation}
This equals the purity of the realigned state and can be measured via the \textbf{swap test} on two copies of $\rho$:

\begin{center}
\begin{quantikz}
\lstick{$\ket{0}$} & \gate{H} & \ctrl{1} & \gate{H} & \meter{} \\
\lstick{$\rho_A$} & \qw & \swap{1} & \qw & \qw \\
\lstick{$\rho_B$} & \qw & \targX{} & \qw & \qw
\end{quantikz}
\end{center}

The probability of measuring $|0\rangle$ on the ancilla gives:
\begin{equation}
P(0) = \frac{1 + \mathrm{Tr}[\rho_A \rho_B]}{2}
\end{equation}

For $\Sigma_2$ of the realignment matrix, we use a \textbf{permuted swap test} where the SWAP acts on rearranged subsystems:
\begin{equation}
\Sigma_2 = 2P_{\text{perm}}(0) - 1
\end{equation}
where the permutation implements the realignment map via qubit routing.

\paragraph{Measuring $G_k$ (realignment eigenvalue moments).}

The eigenvalue moment $G_k = \mathrm{Re}(\mathrm{Tr}[R^k])$ requires measuring traces of powers of the (non-Hermitian) realignment matrix. For $G_2$:
\begin{equation}
G_2 = \mathrm{Re}(\mathrm{Tr}[R^2]) = \mathrm{Re}\left(\sum_{ijkl} \rho_{ij,kl}\rho_{kl,ij}\right)
\end{equation}

This can be measured using a \textbf{cyclic permutation circuit} on two copies:

\begin{center}
\begin{quantikz}
\lstick{$A_1$} & \qw & \swap{2} & \qw & \meter{} \\
\lstick{$B_1$} & \qw & \qw & \qw & \meter{} \\
\lstick{$A_2$} & \qw & \targX{} & \qw & \meter{} \\
\lstick{$B_2$} & \qw & \qw & \qw & \meter{}
\end{quantikz}
\end{center}
The cyclic permutation swaps subsystem $A$ between the two copies while leaving $B$ unchanged.

The cyclic permutation $\Pi_{\text{cyclic}}: A_1 B_1 A_2 B_2 \to A_2 B_1 A_1 B_2$ gives:
\begin{equation}
G_2 = \mathrm{Re}\langle \Pi_{\text{cyclic}} \rangle_{\rho^{\otimes 2}}
\end{equation}

\paragraph{Higher moments $\Sigma_4$, $G_4$.}

For $k=4$, we need \textbf{four copies} of the state and more complex permutation operators:
\begin{align}
\Sigma_4 &= \langle \Pi_S \rangle_{\rho^{\otimes 4}} \\
G_4 &= \mathrm{Re}\langle \Pi_G \rangle_{\rho^{\otimes 4}}
\end{align}
where $\Pi_S$ and $\Pi_G$ are specific permutation operators decomposed into SWAP gates.

The circuit for $\Sigma_4$ measures the 4-cycle permutation on the full systems:
\begin{center}
\begin{quantikz}
\lstick{$\rho^{(1)}$} & \swap{1} & \qw & \qw & \meter{} \\
\lstick{$\rho^{(2)}$} & \targX{} & \swap{1} & \qw & \meter{} \\
\lstick{$\rho^{(3)}$} & \qw & \targX{} & \swap{1} & \meter{} \\
\lstick{$\rho^{(4)}$} & \qw & \qw & \targX{} & \meter{}
\end{quantikz}
\end{center}

For $G_4$, the permutation acts on realigned indices. For a bipartite system $AB$, each copy has subsystems $(A_i, B_i)$:
\begin{center}
\begin{quantikz}
\lstick{$A_1$} & \swap{2} & \qw & \qw & \qw & \meter{} \\
\lstick{$B_1$} & \qw & \swap{2} & \qw & \qw & \meter{} \\
\lstick{$A_2$} & \targX{} & \qw & \swap{2} & \qw & \meter{} \\
\lstick{$B_2$} & \qw & \targX{} & \qw & \swap{2} & \meter{} \\
\lstick{$A_3$} & \qw & \qw & \targX{} & \qw & \meter{} \\
\lstick{$B_3$} & \qw & \qw & \qw & \targX{} & \meter{} \\
\lstick{$A_4$} & \qw & \qw & \qw & \qw & \meter{} \\
\lstick{$B_4$} & \qw & \qw & \qw & \qw & \meter{}
\end{quantikz}
\end{center}
The key difference is that $\Sigma_4$ permutes entire systems cyclically $(1\to 2\to 3\to 4\to 1)$, while $G_4$ permutes subsystems $A$ and $B$ independently in a pattern that corresponds to the realignment map.

\paragraph{Circuit complexity.}

\begin{table}[htb!]
\centering
\caption{Quantum resources for measuring two-feature classifier components.}
\label{tab:two_feature_resources}
\begin{tabular}{@{}lccc@{}}
\toprule
\textbf{Feature} & \textbf{State copies} & \textbf{Circuit depth} & \textbf{Measurements} \\
\midrule
$\Sigma_2$ & 2 & $O(d)$ & 1 \\
$G_2$ & 2 & $O(d)$ & 1 \\
$\Sigma_4$ & 4 & $O(d)$ & 1 \\
$G_4$ & 4 & $O(d)$ & 1 \\
\midrule
$D_4 = \Sigma_4 - G_4$ & 4 & $O(d)$ & 2 \\
$\Sigma_2^2/\Sigma_4$ & 4 & $O(d)$ & 2 \\
\midrule
\textbf{Full classifier} & \textbf{4} & $O(d)$ & \textbf{4} \\
\bottomrule
\end{tabular}
\end{table}

\paragraph{Practical implementation.}

For a $3\times 3$ system (two qutrits, encodable in 4 qubits):
\begin{enumerate}
\item Prepare 4 copies of the unknown state $\rho^{\otimes 4}$.
\item Apply controlled permutation circuits to measure $\Sigma_2$, $\Sigma_4$, $G_2$, $G_4$.
\item Compute $D_4 = \Sigma_4 - G_4$ and $\Sigma_2^2/\Sigma_4$ classically.
\item Apply decision boundary Eq.~\eqref{eq:two_feature_criterion}.
\end{enumerate}

The entire protocol requires only \textbf{4 copies} and \textbf{4 measurement settings}, making it experimentally feasible with current quantum hardware.

%==============================================================================
\subsection{Four-feature RBF SVM classifier}
\label{sec:four_feature_svm}
%==============================================================================

While the two-feature linear classifier (Eq.~\eqref{eq:two_feature_criterion}) achieves perfect separation on the initial training set, testing on $5 \times 10^5$ random separable states reveals a false positive rate of approximately 4\%, due to genuine overlap between separable and bound entangled states in the $(D_4, \Sigma_2^2/\Sigma_4)$ plane. Specifically, low-rank separable states (formed from few product-state terms) can exhibit small $D_4$ values that fall below the linear decision boundary.

To resolve this, we expand the feature space to four dimensions and employ a nonlinear kernel classifier.

\subsubsection{Feature set}

The classifier uses four realignment-matrix spectral features, all measurable via controlled-SWAP circuits on up to four state copies:
\begin{enumerate}
\item $D_4 = \Sigma_4 - G_4$: fourth-order non-Hermiticity gap
\item $D_2 = \Sigma_2 - G_2$: second-order non-Hermiticity gap
\item $G_2 = \mathrm{Re}\,\mathrm{Tr}[R^2]$: second-order eigenvalue moment
\item $\Sigma_2^2/\Sigma_4$: singular value spectral concentration
\end{enumerate}
The additional features $D_2$ and $G_2$ provide complementary second-order information about the realignment matrix structure, resolving ambiguities that arise when only $D_4$ and $\Sigma_2^2/\Sigma_4$ are used.

\subsubsection{Training procedure}

The classifier is a support vector machine (SVM) with Gaussian radial basis function (RBF) kernel:
\begin{equation}
K(\mathbf{x}, \mathbf{x}') = \exp\!\bigl(-\gamma\,\|\mathbf{x} - \mathbf{x}'\|^2\bigr).
\end{equation}
Features are standardized (zero mean, unit variance) before kernel evaluation. The training set consists of:
\begin{itemize}
\item \textbf{50{,}000 random separable states} with varied ranks (1--81 product-state terms), generated with diverse spectral properties to cover the full separable feature space.
\item \textbf{96 bound entangled states} from three structurally distinct families:
  \begin{itemize}
  \item 25 Horodecki states at $a \in \{0.02, 0.06, \ldots, 0.98\}$
  \item 1 Tiles UPB state
  \item 12 chessboard states with distinct parameter sets
  \item 54 noise admixtures: $p\,\rho_{\mathrm{BE}} + (1{-}p)\,\mathbb{I}_9/9$ for $p \in \{0.85, 0.90, 0.95\}$ applied to 18 base states (5 Horodecki, Tiles, and all 12 chessboard variants)
  \item 7 cross-family mixtures (Horodecki--Tiles, chessboard--Tiles, and chessboard--chessboard)
  \end{itemize}
  All derived states were verified as bound entangled via the Carath\'eodory separability gap (Hilbert--Schmidt distance to the nearest separable state estimated by NNLS column generation), retaining 96 of 99 candidates (those with gap $> 10^{-3}$).
\end{itemize}

Hyperparameters were selected by grid search over $C \in \{100, 200, 500, 1000, 2000, 5000\}$ and $\gamma \in \{0.5, 0.8, 1.0, 1.5, 2.0, 3.0, 5.0\}$, requiring 96/96 BE recall and minimizing false positives on 100{,}000 unseen random separable states (generated with a different random seed). Leave-one-out cross-validation (LOO-CV) on a 68-state evaluation set (27 separable + 41 bound entangled from the simulation validation, Section~\ref{sec:be_hardware}) yields 68/68 accuracy at the optimal hyperparameters, confirming that no single state is an outlier driving the decision boundary.

\subsubsection{Performance}

The optimal hyperparameters are $C = 2000$, $\gamma = 3.0$, yielding:
\begin{itemize}
\item \textbf{BE recall:} 96/96 (100\%)
\item \textbf{False positives:} 38 out of 100{,}000 random separable states (\textbf{0.038\%})
\item \textbf{Support vectors:} 121 (61 separable class, 60 BE class)
\end{itemize}

\begin{table}[htb!]
\centering
\caption{Performance comparison of bound entanglement classifiers for $3 \times 3$ systems.}
\label{tab:classifier_comparison}
\begin{tabular}{@{}lcccc@{}}
\toprule
\textbf{Classifier} & \textbf{Features} & \textbf{BE recall} & \textbf{FP rate} & \textbf{Hardware tested} \\
\midrule
$D_2$ threshold & 1 & varies & 0\% & IBM Fez \\
Linear $(D_4, \Sigma_2^2/\Sigma_4)$ & 2 & 100\% & $\sim$4\% & IBM Marrakesh \\
RBF SVM & 4 & 100\% & 0.038\% & Simulation \\
\bottomrule
\end{tabular}
\end{table}

The false positive states are concentrated at low ranks (4--15 product-state terms), where separable states can exhibit small $D_4$ combined with feature values that locally resemble bound entangled signatures. These represent a genuine overlap region in feature space that no classifier based on $(D_4, D_2, G_2, \Sigma_2^2/\Sigma_4)$ alone can fully resolve.

\paragraph{Representativeness of the false positive evaluation.} The $10^5$ random separable test states are generated as convex mixtures of Haar-random product states with varied ranks (1--81 terms), providing broad coverage of the separable state space. This generation procedure emphasises the most experimentally relevant regime: low-rank mixtures (where false positives concentrate) are well-represented, and the rank distribution spans from pure product states to highly mixed convex combinations approaching the maximally mixed state. While structured separable states arising from specific physical processes (e.g., thermal states of local Hamiltonians, or states prepared by shallow circuits) could in principle populate different regions of the feature space, the low-rank regime $r \in [4, 15]$ where all 38 false positives occur is already densely sampled in our test set. Furthermore, the false positive rate decreases with increasing rank: no separable state with $r > 15$ product-state terms is misclassified, suggesting that physically realistic high-rank thermal mixtures are robustly classified.

\subsubsection{Decision function}

For a new state with features $\mathbf{x} = (D_4, D_2, G_2, \Sigma_2^2/\Sigma_4)$, the classifier computes:
\begin{equation}
f(\mathbf{x}) = \sum_{i=1}^{121} \alpha_i \, \exp\!\bigl(-3.0\,\|\tilde{\mathbf{x}} - \tilde{\mathbf{x}}_i\|^2\bigr) + b,
\label{eq:svm_decision}
\end{equation}
where $\tilde{\mathbf{x}} = (\mathbf{x} - \boldsymbol{\mu})/\boldsymbol{\sigma}$ is the standardized feature vector, $\{\tilde{\mathbf{x}}_i\}$ are the 121 support vectors, $\{\alpha_i\}$ are dual coefficients, and $b$ is the intercept. The state is classified as bound entangled when $f(\mathbf{x}) > 0$.

The minimum decision value among all 96 training BE states is $f = 0.9995$ (for a chessboard noise admixture), providing substantial margin. Among the 38 false positive separable states, decision values range from 0.006 to 1.77, with mean 0.58.

\subsubsection{Simulation validation}

The classifier was validated on 68 states computed from exact density matrices (no circuit noise):
\begin{itemize}
\item 27 separable states with varied ranks (2--81 terms)
\item 25 Horodecki states ($a = 0.02, 0.06, \ldots, 0.98$)
\item 1 Tiles UPB state
\item 12 chessboard states
\item 3 verified noise admixtures
\end{itemize}
All 68 states were classified correctly (27/27 SEP, 41/41 BE), confirming $100\%$ accuracy on the test set (Fig.~\ref{fig:simulation_be}).

%==============================================================================
\section{Hardware validation details}
\label{sec:hardware}
%==============================================================================

\subsection{Experimental configuration}

\begin{table}[htb!]
\centering
\caption{IBM Quantum processor specifications.}
\label{tab:hardware_specs}
\begin{tabular}{@{}lcccc@{}}
\toprule
Property & Kingston & Torino & Fez & Marrakesh \\
\midrule
Architecture & Heron & Heron & Heron & Heron \\
Qubits & 156 & 133 & 156 & 156 \\
Native gates & \multicolumn{4}{c}{CZ, $\sqrt{X}$, $R_Z$, $X$} \\
Median $T_1$ ($\mu$s) & 150 & 140 & 150 & 150 \\
Median $T_2$ ($\mu$s) & 100 & 95 & 100 & 100 \\
Median CZ error & 0.5\% & 0.6\% & 0.5\% & 0.5\% \\
\bottomrule
\end{tabular}
\end{table}

All circuits were transpiled using Qiskit optimization level 3. Negativity and chirality experiments (Kingston, Torino) used $10^5$ shots; bound entanglement classification used $4 \times 10^3$ shots per circuit on both Fez ($D_2$ criterion) and Marrakesh (two-feature $D_4$ classifier). Readout errors were mitigated using M3.

\subsection{State preparation}

The parametrized two-qubit states $|\psi(\theta)\rangle = \cos(\theta/2)|00\rangle + \sin(\theta/2)|11\rangle$ were prepared using a standard entangling circuit: (i) apply $R_y(\theta)$ to qubit $A$, producing $\cos(\theta/2)|0\rangle + \sin(\theta/2)|1\rangle$; (ii) apply CNOT from $A$ to $B$, yielding the target state. Bell states correspond to $\theta = 90^\circ$ with appropriate single-qubit corrections.

For $2 \times 3$ experiments, each qutrit was encoded in two physical qubits using the computational subspace $\{|00\rangle, |01\rangle, |10\rangle\} \cong \{|0\rangle, |1\rangle, |2\rangle\}$, with the $|11\rangle$ state excluded by construction. The state $|\psi(\theta)\rangle = \cos(\theta/2)|00\rangle_A|00\rangle_B + \sin(\theta/2)|01\rangle_A|01\rangle_B$ was prepared analogously using $R_y(\theta)$ on the first qubit of subsystem $A$ followed by controlled operations to entangle with subsystem $B$.

For bound entanglement classification on IBM Marrakesh, the $3 \times 3$ Horodecki and separable test states used the same qutrit-in-two-qubit encoding (four physical qubits per copy, 16 data qubits total for four-copy circuits).

\subsection{Statistical uncertainty estimation}

Statistical uncertainties on measured moments were estimated via bootstrap resampling. For each circuit configuration, the $N = 10^5$ measurement outcomes were resampled with replacement $B = 1{,}000$ times. Each bootstrap sample yields a set of moment estimates $\{\mu_k^{(b)}\}_{b=1}^B$, from which the standard error is computed as the standard deviation across resamples:
\begin{equation}
\sigma_{\mu_k} = \mathrm{std}\bigl(\{\mu_k^{(b)}\}_{b=1}^B\bigr).
\end{equation}
Propagating through the Newton--Girard reconstruction and maximum likelihood calibration, the resulting uncertainty on negativity is $\pm 0.005$ for $10^5$ shots, confirming that statistical shot noise is subdominant to systematic gate errors (which produce mean errors of 0.002--0.027 depending on processor and system dimension).

\subsection{Two-stage maximum likelihood calibration}

\textbf{Stage 1 (Calibration):} Using states with known $\theta$, fit degradation factors $(f_2, f_3, f_4)$ for each moment $\mu_k$ by minimizing:
\begin{equation}
\chi^2 = \sum_{i,X} \frac{(X_i^{\mathrm{meas}} - f_X \cdot X_i^{\mathrm{theo}}(\theta_i))^2}{\sigma_X^2}.
\end{equation}

\textbf{Stage 2 (Blind estimation):} For each state, fit $\theta$ using calibrated factors, then compute physical invariants:
\begin{equation}
X^{\mathrm{phys}} = X^{\mathrm{meas}} / f_X.
\end{equation}

\subsection{Calibration results}

\begin{table}[htb!]
\centering
\caption{Two-stage ML calibration parameters.}
\label{tab:calibration}
\begin{tabular}{@{}lccc@{}}
\toprule
System & $f_2$ & $f_3$ & $f_4$ \\
\midrule
Torino $2 \times 2$ & 0.729 & 0.612 & 0.456 \\
Torino $2 \times 3$ & 0.786 & 0.504 & 0.219 \\
\bottomrule
\end{tabular}
\end{table}

\subsection{Calibration sensitivity analysis}

The maximum likelihood calibration assumes that calibration states are prepared at their nominal angles $\theta_i$. To assess robustness to preparation imperfections, we performed a perturbation analysis: systematic offsets $\delta\theta$ were added to all calibration angles, and the entire calibration pipeline (fitting $f_k$ and then estimating negativity) was re-run.

\begin{table}[htb!]
\centering
\caption{Sensitivity of calibration to preparation angle offsets (Torino $2 \times 2$).}
\label{tab:calibration_sensitivity}
\begin{tabular}{@{}lccccc@{}}
\toprule
$\delta\theta$ & $f_2$ & $f_3$ & $f_4$ & Mean $\mathcal{N}$ error & $\Delta f_4 / f_4$ \\
\midrule
$-2^\circ$ & 0.729 & 0.614 & 0.459 & 0.013 & 0.7\% \\
$-1^\circ$ & 0.729 & 0.613 & 0.457 & 0.012 & 0.2\% \\
$0^\circ$ (nominal) & 0.729 & 0.612 & 0.456 & 0.012 & --- \\
$+1^\circ$ & 0.729 & 0.611 & 0.454 & 0.012 & 0.4\% \\
$+2^\circ$ & 0.729 & 0.610 & 0.452 & 0.014 & 0.9\% \\
\bottomrule
\end{tabular}
\end{table}

The results confirm that the calibration is robust: $\pm 2^\circ$ systematic errors in preparation angles change the fitted degradation factors by $<1\%$ and the mean negativity error by $<0.003$. The second-order moment $f_2$ is invariant to small angle perturbations because $\mu_2 = 1$ for all pure states regardless of $\theta$, providing a strong anchor. The fourth-order factor $f_4$ shows the largest sensitivity, but even $\delta\theta = \pm 2^\circ$ (well beyond typical gate calibration accuracy of $<0.5^\circ$) produces negligible degradation in the final negativity estimates.

\subsection{Negativity results}

\begin{table}[htb!]
\centering
\caption{Negativity measurements on IBM Kingston ($2 \times 2$).}
\label{tab:negativity_kingston}
\begin{tabular}{@{}lccc@{}}
\toprule
State & $\mathcal{N}_{\mathrm{theory}}$ & $\mathcal{N}_{\mathrm{ML}}$ & Error \\
\midrule
$|00\rangle$ & 0.000 & 0.000 & 0.000 \\
$\theta = 30^\circ$ & 0.250 & 0.255 & 0.005 \\
$\theta = 45^\circ$ & 0.354 & 0.352 & 0.002 \\
$\theta = 60^\circ$ & 0.433 & 0.437 & 0.004 \\
$\theta = 90^\circ$ & 0.500 & 0.500 & 0.000 \\
\midrule
\textbf{Mean error} & & & \textbf{0.002} \\
\bottomrule
\end{tabular}
\end{table}

\begin{table}[htb!]
\centering
\caption{Negativity measurements on IBM Torino ($2 \times 2$).}
\label{tab:negativity_torino_22}
\begin{tabular}{@{}lccc@{}}
\toprule
State & $\mathcal{N}_{\mathrm{theory}}$ & $\mathcal{N}_{\mathrm{ML}}$ & Error \\
\midrule
$|00\rangle$ & 0.000 & 0.000 & 0.000 \\
$|\Phi^-\rangle$ & 0.500 & 0.500 & 0.000 \\
$|\Phi^+\rangle$ & 0.500 & 0.495 & 0.005 \\
$|\Psi^-\rangle$ & 0.500 & 0.500 & 0.000 \\
$\theta = 15^\circ$ & 0.129 & 0.178 & 0.049 \\
$\theta = 30^\circ$ & 0.250 & 0.239 & 0.011 \\
$\theta = 45^\circ$ & 0.354 & 0.371 & 0.017 \\
$\theta = 60^\circ$ & 0.433 & 0.423 & 0.010 \\
\midrule
\textbf{Mean error} & & & \textbf{0.012} \\
\bottomrule
\end{tabular}
\end{table}

\begin{table}[htb!]
\centering
\caption{Negativity measurements on IBM Torino ($2 \times 3$).}
\label{tab:negativity_torino_23}
\begin{tabular}{@{}lccc@{}}
\toprule
State & $\mathcal{N}_{\mathrm{theory}}$ & $\mathcal{N}_{\mathrm{ML}}$ & Error \\
\midrule
$\theta = 0^\circ$ & 0.000 & 0.000 & 0.000 \\
$\theta = 15^\circ$ & 0.129 & 0.107 & 0.022 \\
$\theta = 30^\circ$ & 0.250 & 0.189 & 0.061 \\
$\theta = 45^\circ$ & 0.354 & 0.335 & 0.019 \\
$\theta = 60^\circ$ & 0.433 & 0.405 & 0.028 \\
$\theta = 90^\circ$ & 0.500 & 0.469 & 0.031 \\
\midrule
\textbf{Mean error} & & & \textbf{0.027} \\
\bottomrule
\end{tabular}
\end{table}

\subsection{Chirality correction results}

\begin{table}[htb!]
\centering
\caption{Chirality correction measurements on IBM Torino.}
\label{tab:chirality}
\begin{tabular}{@{}l|ccc|ccc@{}}
\toprule
& \multicolumn{3}{c|}{$2 \times 2$} & \multicolumn{3}{c}{$2 \times 3$} \\
$\theta$ & $(-C_4)_{\mathrm{theory}}$ & $(-C_4)_{\mathrm{ML}}$ & Error & $(-C_4)_{\mathrm{theory}}$ & $(-C_4)_{\mathrm{ML}}$ & Error \\
\midrule
$0^\circ$ & 0.000 & 0.000 & 0.000 & 0.000 & 0.000 & 0.000 \\
$15^\circ$ & 0.066 & 0.122 & 0.056 & 0.066 & 0.045 & 0.021 \\
$30^\circ$ & 0.234 & 0.215 & 0.019 & 0.234 & 0.138 & 0.096 \\
$45^\circ$ & 0.438 & 0.465 & 0.027 & 0.438 & 0.425 & 0.013 \\
$60^\circ$ & 0.609 & 0.580 & 0.029 & 0.609 & 0.505 & 0.104 \\
$90^\circ$ & 0.750 & 0.750 & 0.000 & 0.750 & 0.750 & 0.000 \\
\midrule
\textbf{Mean} & & & \textbf{0.022} & & & \textbf{0.039} \\
\bottomrule
\end{tabular}
\end{table}

\subsection{Bound entanglement detection on IBM hardware}
\label{sec:be_hardware}

We validated bound entanglement detection on two IBM processors. The simplified $D_2$ criterion was tested on IBM Fez (Table~\ref{tab:be_fez}), and the two-feature classifier (Eq.~\eqref{eq:two_feature_criterion}) was tested on IBM Marrakesh (Table~\ref{tab:be_hardware}). Both experiments used the same four $3 \times 3$ states: two random separable mixtures and two Horodecki bound entangled states~\cite{Horodecki1997} with parameters $a = 0.30$ and $a = 0.70$ (Eq.~\eqref{eq:horodecki_state_si}). Each qutrit was encoded in two qubits. On Fez, the two-copy circuits for $\Sigma_2$ and $G_2$ measurement used 8 data qubits; on Marrakesh, the four-copy circuits for $\Sigma_4$ and $G_4$ used 16 data qubits. All circuits were executed with 4{,}000 shots.

\begin{table}[htb!]
\centering
\caption{Bound entanglement detection on IBM Fez using the simplified $D_2 = \Sigma_2 - G_2$ criterion. States with $D_2 < 0.127$ are classified as bound entangled.}
\label{tab:be_fez}
\begin{tabular}{@{}lccccl@{}}
\toprule
\textbf{State} & $\Sigma_2$ & $G_2$ & $D_2$ & Predicted & True \\
\midrule
Separable 1 & 0.267 & 0.136 & 0.131 & SEP & SEP \\
Separable 2 & 0.330 & 0.146 & 0.184 & SEP & SEP \\
Horodecki ($a\!=\!0.30$) & 0.216 & 0.116 & 0.100 & BE & BE \\
Horodecki ($a\!=\!0.70$) & 0.197 & 0.121 & 0.076 & BE & BE \\
\midrule
\multicolumn{5}{l}{\textbf{Classification accuracy}} & \textbf{4/4} \\
\bottomrule
\end{tabular}
\end{table}

\begin{table}[htb!]
\centering
\caption{Bound entanglement detection on IBM Marrakesh. The two-feature classifier $D_4 < -0.026 + 0.022 \cdot (\Sigma_2^2/\Sigma_4)$ correctly identifies all four $3\times 3$ states. The threshold column gives the right-hand side of Eq.~\eqref{eq:two_feature_criterion} evaluated at the measured $\Sigma_2^2/\Sigma_4$; a state is classified as bound entangled when $D_4$ falls below this threshold.}
\label{tab:be_hardware}
\begin{tabular}{@{}lccccl@{}}
\toprule
\textbf{State} & $D_4$ & $\Sigma_2^2/\Sigma_4$ & Threshold & Predicted & True \\
\midrule
Separable 1 & 0.036 & 1.84 & 0.014 & SEP & SEP \\
Separable 2 & 0.053 & 2.10 & 0.020 & SEP & SEP \\
Horodecki ($a\!=\!0.30$) & 0.006 & 3.25 & 0.046 & BE & BE \\
Horodecki ($a\!=\!0.70$) & $2.7\!\times\!10^{-4}$ & 5.07 & 0.086 & BE & BE \\
\midrule
\multicolumn{5}{l}{\textbf{Classification accuracy}} & \textbf{4/4} \\
\bottomrule
\end{tabular}
\end{table}

The results confirm the classifier's robustness to hardware noise. Separable states exhibit large $D_4$ values (strong non-Hermiticity of the realignment matrix), placing them well above the decision boundary. Bound entangled states show $D_4 \approx 0$ (nearly Hermitian realignment matrices) combined with high $\Sigma_2^2/\Sigma_4$ (spread singular value spectra), placing them below the boundary with clear margin. The Horodecki state with $a = 0.70$ has $D_4 = 2.7 \times 10^{-4}$, confirming that the near-Hermiticity signature survives realistic hardware noise.

Complementary validation on a FakeMarrakesh noisy simulator (4{,}000 shots, including a Tiles UPB state~\cite{Bennett1999}) also achieved 4/4 classification accuracy, confirming that the decision boundary trained on ideal distributions generalizes to noisy measurement statistics without recalibration. An improved four-feature RBF SVM classifier (Section~\ref{sec:four_feature_svm}), validated in theory-only simulation across 68 states (27 separable, 41 bound entangled from three families plus noise admixtures), achieves $68/68$ classification accuracy with a false positive rate of 0.038\% on $10^5$ random separable states.

\begin{figure}[htb!]
\centering
\includegraphics[width=\textwidth,height=0.55\textheight,keepaspectratio]{fig_simulation_negativity_chirality.pdf}
\caption{\textbf{Simulator validation of negativity and chirality for pure and mixed two-qubit states.}
\textbf{a},~\textbf{b},~Pure parametrised states $|\psi(\theta)\rangle = \cos(\theta/2)|00\rangle + \sin(\theta/2)|11\rangle$ sampled at $2.5^\circ$ steps (37 angles). \textbf{a},~Negativity $\mathcal{N}$ versus $\theta$. \textbf{b},~Chirality $-C_4 = I_4 - \mu_4$ versus $\theta$ (for pure states $I_4 = 1$). Solid line: exact theory; open circles: ideal simulator (shot noise only); filled circles: noisy simulator using IBM Torino calibration data.
\textbf{c},~\textbf{d},~Mixed Werner states $\rho_W(p) = p|\Psi^-\rangle\langle\Psi^-| + (1-p)\mathbb{I}/4$ at $p \in \{0, 0.2, 1/3, 0.5, 0.6, 0.8, 1\}$. \textbf{c},~Negativity versus $p$; dotted line marks the separability threshold $p = 1/3$. \textbf{d},~Chirality $-C_4 = I_4 - \mu_4$ versus $p$; dotted line marks the separability threshold $p = 1/3$.
\textbf{e},~\textbf{f},~Bell-product mixtures $\rho_{\mathrm{BP}}(p) = p|\Psi^-\rangle\langle\Psi^-| + (1-p)|00\rangle\langle 00|$ (entangled for all $p > 0$).
All panels: $8{,}192$ shots per circuit on AerSimulator with Torino noise model (thermal relaxation, depolarising, and readout errors from device calibration).}
\label{fig:simulation_nc}
\end{figure}

\begin{figure}[htb!]
\centering
\includegraphics[width=0.7\textwidth]{fig_simulation_be_classification.pdf}
\caption{\textbf{Bound entanglement classification across 68 qutrit--qutrit ($3 \times 3$) states.}
Four-feature SVM classifier applied to 27 random separable states (blue) and 41 bound entangled states (red) from three structurally distinct families: 25 Horodecki states ($\times$), Tiles UPB ($\diamondsuit$), 12 chessboard parameter sets ($\square$), and verified noise admixtures ($\triangle$). \textbf{a},~Projection onto $(\Sigma_2^2/\Sigma_4,\; D_4)$ plane with SVM decision contour (dashed). \textbf{b},~Projection onto $(G_2,\; D_2)$ plane. Classification accuracy: $68/68$ ($100\%$). The classifier was trained on $5 \times 10^4$ random separable states and 96 known bound entangled states, achieving a false positive rate of $0.038\%$ on $10^5$ unseen separable states.}
\label{fig:simulation_be}
\end{figure}

%==============================================================================
\section{Noise analysis}
\label{sec:noise}
%==============================================================================

\subsection{Depolarizing noise model}

Under depolarizing noise:
\begin{equation}
\rho_{\mathrm{noisy}} = (1-\eta)\rho + \frac{\eta}{d}\mathbb{I}_d.
\end{equation}

For pure target states, the measured purity relates to noise by:
\begin{equation}
I_2^{\mathrm{noisy}} \approx 1 - \frac{3\eta}{2} \quad \text{(for two-qubit systems)}.
\end{equation}

\subsection{RMSE scaling}

Monte Carlo simulation ($N = 10^5$ Haar-random states) yields:
\begin{align}
\mathrm{RMSE}_{2\times2} &= (0.245 \pm 0.004)\eta, \quad R^2 = 0.998, \\
\mathrm{RMSE}_{2\times3} &= (0.219 \pm 0.002)\eta, \quad R^2 = 0.999.
\end{align}

Within typical NISQ noise ($\eta \approx 0.04$--$0.08$), RMSE remains below 0.022.

\subsection{Bias direction}

Depolarizing noise consistently \emph{underestimates} entanglement:
\begin{equation}
\mathcal{N}(\rho_{\mathrm{noisy}}) \approx (1-\eta)\mathcal{N}(\rho).
\end{equation}

This ensures the protocol never falsely reports entanglement in separable states.

%==============================================================================
\section{Efficiency verification}
\label{sec:efficiency}
%==============================================================================

\subsection{Monte Carlo results}

For Haar-random states ($N = 10^5$):

\begin{table}[htb!]
\centering
\caption{Measurement efficiency versus tomography.}
\label{tab:efficiency}
\begin{tabular}{@{}lccc@{}}
\toprule
System & Tomography & This work & Efficiency \\
\midrule
$2 \times 2$ & 16 & 3 & $5.3\times$ \\
$2 \times 3$ & 36 & 5 & $7.2\times$ \\
\bottomrule
\end{tabular}
\end{table}

The degeneracy conditions $\mathcal{G}_1 = 0$ and $\mathcal{G}_2 = 0$ are satisfied with probability zero for uniformly sampled states, so all generic states require the maximum number of moments.

\subsection{Classification distribution}

Among uniformly sampled two-qubit states:
\begin{itemize}
\item 95.3\% have four distinct eigenvalues ($\mathcal{D} \neq 0$)
\item 4.7\% exhibit simple pair degeneracy ($\mathcal{D} = 0$, but $\mathcal{G}_1 \neq 0$, $\mathcal{G}_2 \neq 0$)
\item Both cases require 3 measurements
\end{itemize}

Only specially structured states (Bell, Werner, maximally mixed) satisfy $\mathcal{G}_2 = 0$ and benefit from simplified formulas.

%==============================================================================
\section*{References}
%==============================================================================

\begin{thebibliography}{99}

\bibitem{PhysRevLett.77.1413}
Peres, A. Separability criterion for density matrices. \textit{Phys. Rev. Lett.} \textbf{77}, 1413--1415 (1996).

\bibitem{HORODECKI19961}
Horodecki, M., Horodecki, P. \& Horodecki, R. Separability of mixed states: necessary and sufficient conditions. \textit{Phys. Lett. A} \textbf{223}, 1--8 (1996).

\bibitem{Werner1989}
Werner, R. F. Quantum states with Einstein-Podolsky-Rosen correlations admitting a hidden-variable model. \textit{Phys. Rev. A} \textbf{40}, 4277--4281 (1989).

\bibitem{Verstraete2001}
Verstraete, F., Audenaert, K., Dehaene, J. \& De Moor, B. A comparison of the entanglement measures negativity and concurrence. \textit{J. Phys. A} \textbf{34}, 10327--10332 (2001).

\bibitem{Horodecki1998bound}
Horodecki, M., Horodecki, P. \& Horodecki, R. Mixed-state entanglement and distillation: Is there a ``bound'' entanglement in nature? \textit{Phys. Rev. Lett.} \textbf{80}, 5239--5242 (1998).

\bibitem{Horodecki1997}
Horodecki, P. Separability criterion and inseparable mixed states with positive partial transposition. \textit{Phys. Lett. A} \textbf{232}, 333--339 (1997).

\bibitem{Bennett1999}
Bennett, C. H. \textit{et al.} Unextendible product bases and bound entanglement. \textit{Phys. Rev. Lett.} \textbf{82}, 5385--5388 (1999).

\bibitem{DiVincenzo2000}
DiVincenzo, D. P. \textit{et al.} Evidence for bound entangled states with negative partial transpose. \textit{Phys. Rev. A} \textbf{61}, 062312 (2000).

\bibitem{Lewenstein2000}
Lewenstein, M., Kraus, B., Cirac, J. I. \& Horodecki, P. Optimization of entanglement witnesses. \textit{Phys. Rev. A} \textbf{62}, 052310 (2000).

\bibitem{Wootters1998}
Wootters, W. K. Entanglement of formation of an arbitrary state of two qubits. \textit{Phys. Rev. Lett.} \textbf{80}, 2245--2248 (1998).

\bibitem{Bruss2000chess}
Bru\ss, D. \& Peres, A. Construction of quantum states with bound entanglement. \textit{Phys. Rev. A} \textbf{61}, 030301(R) (2000).

\end{thebibliography}

\end{document}
